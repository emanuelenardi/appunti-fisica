\section{Dinamica del punto}
	\subsection{Legge d'inerzia}
	\begin{itemize}
		\item Legge di Newton $\overrightarrow{F} = m \overrightarrow{a}$
		\item $\overrightarrow{F} = m\frac{d\overrightarrow{V}}{dt} = m\frac{d^2\overrightarrow{r}}{dt^2}$
		\item $\overrightarrow{F} = \bigl(\frac{dm}{dt}\bigr)v+ma$
	\end{itemize}
	\subsection{Quantit\`a di moto}
	\begin{itemize}
		\item Quantit\`a di moto $\overrightarrow{p} = m\overrightarrow{v}$
		\item Legge di Newton $\overrightarrow{F} = \frac{d\overrightarrow{p}}{dt}$
		\item Teorema dell'impulso $\overrightarrow{I} = \Delta\overrightarrow{p}$ con $F$ costante $\overrightarrow{F}t = m(\overrightarrow{v} - \overrightarrow{v}_0$
	\end{itemize}
	\subsection{Forza peso}
	\begin{itemize}
		\item $g = 9.81\frac{m}{s^2}$
		\item $F_p = mg$
	\end{itemize}
	\subsection{Forza di attrito radente}
	\begin{itemize}
		\item $F = \mu_s N$ con $N$ modulo della componente normale al piano di appoggio.
		\item $F_d = \mu_dN$
	\end{itemize}
	\subsection{Forza elastica}
	\begin{itemize}
		\item $\overrightarrow{F_{el}} = -kx\hat{x}$ $k$ costante elastica, $\hat{x}$ versore della differenza di lunghezza rispetto a $O$.
		\item $a = -\frac{k}{m}x = -\omega^2 x$, moto armonico con pulsazione $\omega = \sqrt{\frac{k}{m}}$ $T = 2\pi\sqrt{\frac{m}{k}}$
		\item Equazione differenziale dell'oscillatore armonico semplice $m\frac{d^2x}{dt^2}+kx = 0$ $x(t) = A\cos(\omega t +\phi)$ $v = -A\omega\sin(\omega t+\phi)$ $a =-A\omega^2\cos(\omega t+\phi)$
	\end{itemize}
	\subsection{Piano inclinato}
	\begin{itemize}
		\item $\alpha$ ampiezza del piano $\mu_s$ $\mu_d$ attriti. Asse con $x$ parallelo al piano.
		\item Corpo di massa $m$ scomporre $F_p$ in $F_{px}$ e $F_{py}$
		\item Attrito bilancia $F_{px}$ equilibrio statico $\tan\alpha\le \mu_s$
	\end{itemize}
	\subsection{Forze centripete}
	\begin{itemize}
		\item $F_N$ accelerazione centripeta $F = ma_N = m\frac{v^2}{r}$
	\end{itemize}
	\subsection{Pendolo semplice}
	\begin{itemize}
		\item $T_F = mg$ a equilibrio statico.
		\item $R_T = -mg\sin\theta = ma_T$
		\item $R_N = T_F -mg\cos\theta = ma_N$
		\item Piccoli valori di $\theta$ $\frac{d^2\theta}{dt^2} + \frac{g}{L}\theta = 0$
		\item $\omega^2 = \frac{g}{L}$
		\item $\theta(t)= \theta_0\sin(\omega t+\phi)$
		\item $T = \frac{2\pi}{\omega} = 2\pi\sqrt{\frac{L}{g}}$
		\item $s = L\theta = L\theta_o\sin(\omega t+\phi)$
		\item $v = L\omega\theta_0\cos(\omega t+\phi)$
		\item $\omega(t) = \omega\theta_0\cos(\omega t+\phi)$
		\item $T_F = m\bigl[g\cos(\theta(t)) + \frac{v^2(t)}{L}\bigr]$
	\end{itemize}
	\subsection{Lavoro potenza energia cinetica}
	\begin{itemize}
		\item $W = \int_A^BF_Tds$
		\item $P = \frac{dW}{dt} = \overrightarrow{F}\cdot\overrightarrow{v} = F_Tv$
		\item $E_k = \frac{1}{2}mv^2$
		\item $W = \Delta E_k$
		\item Forze conservative $E_m = E_p + E_k = costante$
		\item Dissipative $W = E_{m,B} - E_{m,A}$
	\end{itemize}
