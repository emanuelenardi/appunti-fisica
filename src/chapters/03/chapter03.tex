\documentclass[class=book, crop=false, oneside, 12pt]{standalone}
\usepackage{standalone}
\usepackage{amsmath}
\usepackage{../../style}

\begin{document}
\chapter{Moti relativi}

\section{Sistemi di riferimento. Velocità e Accelerazione relative}

Sperimentalmente è provato con estrema accuratezza che le leggi fisiche non dipendono dalla scelta del sistema di riferimento. 
Fissato un sistema di riferimento e stabilita una certa proprietà, questa resta vera anche se cambiano l'origine e l'orientazione degli assi coordinati, ovvero se ci riferiamo ad un altro sistema ottenuto dal primo con una traslazione (spostamento dell'origine, conservando la stessa direzione degli assi) o con una rotazione (stessa origine, cambio della direzione degli assi) o con una operazione combinata.
Non esiste pertanto un punto privilegiato dello spazio e nemmeno un'orientazione privilegiata: lo spazio appare omogeneo e isotropo. 

Abbiamo già rilevato come il concetto stesso di moto sia relativo,  abbia cioè bisogno della precisazione del sistema di riferimento. 
% TODO add figura (3.1)
Consideriamo il problema riferendoci alla figura 3.1. 
Il punto \(P\) è in movimento lungo una generica traiettoria. 
Il suo moto viene osservato da una terna cartesiana con centro in \(O\) che, per convenzione, chiamiamo sistema fisso e da una terna cartesiana con centro \(O'\) che, sempre per convenzione, chiamiamo sistema mobile.

Vogliamo ricavare una relazione tra la posizione, la velocità e l'accelerazione del punto \(P\), misurate da un osservatore solidale con il sistema fisso, e le corrispondenti grandezze misurate da un osservatore solidale con il sistema mobile. 

\subsection{Posizione}

La relazione tra le posizioni del punto \(P\), misurate rispetto ai due sistemi di riferimento, è la seguente:
\begin{equation}
    \overrightarrow{r} = \overrightarrow{OO'} + \overrightarrow{r'}
\end{equation}
con 
\begin{equation*}
    \overrightarrow{r} = x \overrightarrow{u_x} + y \overrightarrow{u_y} + z \overrightarrow{u_z} \ , \ \overrightarrow{r'} = x' \overrightarrow{u_{x'}} + y' \overrightarrow{u_{y'}} + z' \overrightarrow{u_{z'}} 
\end{equation*}
Assumiamo, dato che il primo sistema è fisso, che i versori \(\overrightarrow{u_x}, \overrightarrow{u_y}, \overrightarrow{u_z}\) sono indipendenti dal tempo.

\subsection{Velocità}

La velocità del punto \(P\) rispetto al sistema fisso, che chiamiamo velocità assoluta, è data da:
\begin{equation}
    \overrightarrow{v} = \frac{d \overrightarrow{r}}{dt} = \frac{dx}{dt} \overrightarrow{u_x} + \frac{dy}{dt} \overrightarrow{u_y} + \frac{dz}{dt} \overrightarrow{u_z}
\end{equation}
mentre quella misurata da un osservatore nel sistema mobile, che indichiamo come velocità relativa è
\begin{equation}
    \overrightarrow{v'} = \frac{dx'}{dt} \overrightarrow{u_{x'}} +\frac{dy'}{dt} \overrightarrow{u_{y'}} + \frac{dz'}{dt} \overrightarrow{u_{z'}}
\end{equation}
Infine la velocità dell origine \(O'\) del sistema di riferimento mobile misurata da un osservatore del sistema fisso e data da:
\begin{equation}
    \overrightarrow{v_{O'}}=\frac{d \overrightarrow{OO'}}{d t}=\frac{d x_{O'}}{d t} \overrightarrow{u_{x}}+\frac{d y_{O'}}{d t} \overrightarrow{u_{y}}+\frac{d z_{O'}}{d t} \overrightarrow{u_{z}}
\end{equation}

Derivando rispetto il tempo la (3.1) ottengo

\begin{align*}
    \overrightarrow{v}&=\frac{d \overrightarrow{r}}{dt} = \frac{d \overrightarrow{OO'}}{d t} + \frac{d \overrightarrow{r'}}{dt} = \frac{d x_{O'}}{d t} \overrightarrow{u_{x}}+\frac{d y_{O'}}{d t} \overrightarrow{u_{y}}+\frac{d z_{O'}}{d t} \overrightarrow{u_{z}}\\
    &+\frac{dx'}{dt} \overrightarrow{u_{x'}} + \frac{dy'}{dt} \overrightarrow{u_{y'}} + \frac{dz'}{dt} \overrightarrow{u_{z'}} + x' \frac{d\overrightarrow{u_{x'}}}{dt}  + y' \frac{d\overrightarrow{u_{y'}}}{dt}  + z' \frac{d\overrightarrow{u_{z'}}}{dt}\\
\end{align*}
ovvero
\begin{equation}
    \overrightarrow{v} = \overrightarrow{v_{O'}} + \overrightarrow{v'} + x' \frac{d\overrightarrow{u_{x'}}}{dt}  + y' \frac{d\overrightarrow{u_{y'}}}{dt}  + z' \frac{d\overrightarrow{u_{z'}}}{dt}
\end{equation}

La derivata di un versore \(\overrightarrow{u}\), in quanto vettore con modulo costante, si può scrivere \(\omega \times \overrightarrow{u}\) ; pertanto per le derivate dei tre versori \(\overrightarrow{u_x},\overrightarrow{u_y},\overrightarrow{u_z}\), si hanno le seguenti formule, dette di Poisson:
\begin{equation}
    \frac{d \overrightarrow{u_{x}}}{d t}=\omega \times \overrightarrow{u_{x}} \quad, \quad \frac{d \mathbf{u_{y'}}}{d t}=\omega \times \overrightarrow{u_{y}}, \quad, \quad \frac{d \overrightarrow{u_{z'}}}{d t}=\omega \times \overrightarrow{u_{z}}
\end{equation} 

Posso riscrivere gli ultimi temini dell'equazione della velocità come:
\begin{equation}
    x' (\omega \overrightarrow{u_{x'}}) + y' (\omega \overrightarrow{u_{y'}}) + z' (\omega \overrightarrow{u_{z'}}) = \omega \times (x' \overrightarrow{u_{x'}} + y' \overrightarrow{u_{y'}} + z' \overrightarrow{u_{z'}}) = \omega \times \overrightarrow{r'}
\end{equation}

\subsubsection{Teorema delle velocità relative}
Effettuando l'ultima sostituzione ottengo infine:

\begin{equation}
    \overrightarrow{v} = \overrightarrow{v_{O'}} + \overrightarrow{v'} + \omega \times \overrightarrow{r'}
\end{equation}

La differenza delle due velocità misurate nei sistemi di riferimento è chiamata \emph{velocità di trascinamento}
\begin{equation*}
    \overrightarrow{v_t} = \overrightarrow{v} - \overrightarrow{v'} = \overrightarrow{v_{O'}} + \omega \times \overrightarrow{r'}
\end{equation*}
La velocità di trascinamento è la velocità che il punto mobile \(P\) avrebbe se, nell’istante considerato, fosse solidale con il sistema relativo.

Il moto di trascinamento, legato in pratica al moto del sistema mobile. può essere considerato in ogni istante come la somma di un termine traslatorio con velocità istantanea \(v_{O'}\) e di un termine rotatorio con velocità angolare \(\omega\), variabile in generale sia in modulo che in direzione.

\subsection{Accelerazione}

L'acceleraizone rispetto al sistema fisso viene detta \emph{accelerazione assoluta} ed è pari:

\begin{equation}
    \overrightarrow{a} = \frac{d^2 x }{dt^2} \overrightarrow{u_x} + \frac{d^2 y }{dt^2} \overrightarrow{u_y} + \frac{d^2 z }{dt^2} \overrightarrow{u_z}
\end{equation}
mentre rispetto al sistema mobile l'\emph{accelerazione relativa} è
\begin{equation}
    \overrightarrow{a'} = \frac{d^2 x' }{dt^2} \overrightarrow{u_{x'}} + \frac{d^2 y' }{dt^2} \overrightarrow{u_{y'}} + \frac{d^2 z' }{dt^2} \overrightarrow{u_{z'}} . 
\end{equation}
L'accelerazione del sistema mobile \(O'\) rispetto a \(O\) è dato da \(\overrightarrow{a_{0'}} = \frac{d \overrightarrow{v_{O'}}}{dt}\). Derivando rispetto al tempo ottengo:
\begin{equation}
    \overrightarrow{a} = \frac{d \overrightarrow{v}}{dt} = \frac{d \overrightarrow{v_{O'}}}{dt} + \frac{d \overrightarrow{v'}}{dt} + \frac{d \overrightarrow{\omega}}{dt} \times \overrightarrow{r'} + \omega \times \frac{d \overrightarrow{r'}}{dt}
\end{equation}
Calcolando \(d \overrightarrow{v'} / dt\) ricaviamo:
\begin{equation}
\begin{aligned}
    \frac{d \overrightarrow{v'}}{d t} &=\frac{d}{d t} \left(\frac{d x'}{d t} \overrightarrow{u}_{x'}+\frac{d y'}{d t} \overrightarrow{u}_{y'}+\frac{d z'}{d t} \overrightarrow{u}_{z'}\right) = \left( \frac{d^{2} x'}{d t^{2}} \overrightarrow{u}_{x'}+\frac{d^{2} y'}{d t^{2}} \overrightarrow{u}_{y'} + \frac{d^{2} z'}{d t^{2}} \overrightarrow{u}_{z'} \right)\\
    & + \left( \frac{d x'}{d t} \frac{d \overrightarrow{u}_{x'}}{d t}+\frac{d y'}{d t} \frac{d \overrightarrow{u}_{y'}}{d t}+\frac{d z'}{d t} \frac{d \overrightarrow{u}_{z'}}{d t} \right)=\overrightarrow{a'}+\omega \times \overrightarrow{v'} 
\end{aligned}
\end{equation}
Ho inoltre:
\begin{equation*}
    \omega \times \frac{d \overrightarrow{r'}}{dt} = \omega \times \overrightarrow{v'} + \omega \times ( \omega \times \overrightarrow{r'})
\end{equation*}

\subsubsection{Teorema delle velocità relative}

Ottengo quindi infine:

\begin{equation}
    \overrightarrow{a} = \overrightarrow{a'} + \overrightarrow{a_{O'}} + \omega \times ( \omega \times \overrightarrow{r'}) + \frac{d \omega}{dt} \times \overrightarrow{r'} + 2 \omega \times \overrightarrow{v'}
\end{equation}

Le accelerazioni del punto \(P\) misurate nei due sistemi non coincidono ma sono messe in relazione tramite la (3.12), detta \emph{teorema delle accelerazioni relative}.
Per valutare l'accelerazione di trascinamento \(\overrightarrow{a}\), riprendiamo la discussione fatta per la velocità di trascinamento. 
L'accelerazione di trascinamento è quella del punto \(P^{\star}\), solidale col sistema mobile, che coincide nell'istante considerato col punto \(P\). 
Per \(P^{\star}, \overrightarrow{a', v'}\) sono nulle e da (3.12)
\begin{equation}
    \overrightarrow{a_t} = \overrightarrow{a_{O'}} + \omega \times (\omega \times \overrightarrow{r'}) + \frac{d \omega }{dt} \times \overrightarrow{r'}
\end{equation}

Possiamo riscrivere (3.12) come:
\begin{equation}
    \overrightarrow{a} = \overrightarrow{a'} + \overrightarrow{a_t} + \overrightarrow{a_c}
\end{equation}
l'ultimo termine
\begin{equation*}
    \overrightarrow{a_c} = 2 \omega \times \overrightarrow{v'}
\end{equation*}
è chiamato accelerazione complementare o di Coriolis; esso dipende dal moto di \(P\) rispetto al sistema mobile tramite la velocità relativa di \(\overrightarrow{v'}\)

\section{Sistemi di riferimento inerziali, Relatività Galileiana}

\subsection{Sistema di riferimento inerziale}

Definiamo come sistema di riferimento inerziale un sistema in cui valga rigorosamente la legge di inerzia, in cui cioè un punto non soggetto a f or:e la nc iato con ve locità arbitraria ip qualunque direzione si muo va con moto re ttilineo uniforme o, se è  in quiete, resti  in quie te.
È evidente che la definizione di sistema di riferimento inerziale ha significato solo se siamo in grado di verificare in modo diverso che il punto non è soggetto a forze. 
È ragionevole supporre che questa situazione si verifichi sia quando il punto è sufficientemente lontano da ogni altro corpo in modo da poter trascurare  ogni interazione, sia quando è possibile bilanciare le forze agenti in modo che la risultante sia nulla.

In un sistema di riferimento inerziale la legge di Newton ha l'espressione più semplice: 
le forze che compaiono a primo membro sono le forze vere cioè quelle che sappiamo derivare dalle interazioni fondamentali e la risultante è proporzionale all'accelerazione  misurata in quel sistema di riferimento. 

\subsection{Relatività galileiana}

Consideriamo ora un altro sistema di riferimento che si muove di moto traslatorio rettilineo uniforme rispetto ad un certo sistema inerziale. 
Pertanto si ha 
\begin{equation}
    \overrightarrow{v_{O'}} = costante \ , \ \overrightarrow{a_{O'}} = 0 \ e \ \omega = 0
\end{equation}
Da ricaviamo \(\overrightarrow{a} = \overrightarrow{a'}\): le accelerazioni di un punto misurate nei due sistemi di riferimento sono eguali. 
Se \(\overrightarrow{a} = 0\) anche \(\overrightarrow{a'}= 0\) e quindi il secondo sistema è pure inerziale. 

Abbiamo così ottenuto questo risultato fondamentale: \emph{definito un sistema di riferimento inerziale, tutti gli altri sistemi in altri sistemi in moto rettilineo uniforme rispetto a questo sono anch'essi inerziali};
per tali sistemi la legge di Newton si scrive allo stesso modo, ossia con gli stessi valori di \(\overrightarrow{F}\) e di \(\overrightarrow{a}\); in particolare se \(\overrightarrow{a} = 0\) in uno, essa è nulla in tutti gli altri.

Conseguenza importante è che, essendo la dinamica la stessa non è possibile stabilire, tramite misure effettuate in questi diversi sistemi di riferimento, se uno di essi è in quiete o in moto. 
Non ha cioè senso il concetto di moto assoluto. 
Tale situazione fisica viene descritta anche con il termine di \emph{relatività galileiana}.

Se il moto del secondo sistema è accelerato rispetto al sistema inerziale, (\(a_{O'} \neq 0\) oppure \(\omega \neq 0\) o entrambe) si osserva che la legge di Newton non è più valida, la forza vera che agisce sul punto considerato non è proporzionale all'accelerazione del punto, misurata nel sistema accelerato.

Infatti, se \(\overrightarrow{F} = m \overrightarrow{a}\) nel sistema inerziale, nel sistema mobile in moto accelerato non può valere \(\overrightarrow{F} = m \overrightarrow{a'}\) poiché \( \overrightarrow{a'} \neq \overrightarrow{a}\).
Ho anche che, se moltiplichiamo i termini di (3.15) per la massa del punto e teniamo conto che \(\overrightarrow{F} = m \overrightarrow{a}\), abbiamo:
\begin{equation*}
    \overrightarrow{F} - m \overrightarrow{a_t} - m \overrightarrow{a_c} = m \overrightarrow{a'}
\end{equation*}

\subsection{Forze apparenti o forze di inerzia}
Questa equazione rappresenta una forma modificata dalla legge di Newton: in un sistema non inerziale il prodotto della massa del punto materiale per l'accelerazione misurata in quel sistema è eguale alla forza vera agente sul punto più le forze apparenti. 
Queste ultime forze, che sono sempre proporzionali alla massa del punto e vengono pertanto chiamate anche forze di inerzia, appaiono agenti solo nel sistema non inerziale, dove costituiscono il termine correttivo che permette di ritornare ad una espressione \(\overrightarrow{F'} = m \overrightarrow{a'}\).
Le forze apparenti non derivano dalle interazioni fondamentali e non esistono in un sistema di riferimento inerziale.

\subsection{Sistema non inerziale}

Se osserviamo in un sistema inerziale un punto materiale che descrive una traiettoria curva possiamo affermare che su di esso agisce una forza (vera); se \(\overrightarrow{F} = 0\) sappiamo che il moto è rettilineo uniforme e viceversa.\newline
In un sistema accelerato vediamo da (3.15) che \(\overrightarrow{F} = 0\) non comporta \(\overrightarrow{a}' = 0\) e quindi l'osservazione di un moto rettilineo uniforme. 
Questo risultato giustifica il nome di sistema non inerziale per un sistema accelerato. 
Analogamente, una traiettoria curva non presuppone necessariamente l'azione di una forza (vera), ma può essere un effetto apparente, conseguenza del moto accelerato del sistema in cui si trova l'osservatore, e così vi a.

In entrambi i sistemi, note le condizioni iniziali del moto e le forze agenti, facciamo previsioni corrette per il moto di un punto tramite l'equazione di newton e il teorema delle accelerazioni relative. 
Però nel sistema non inerziale la descrizione è più complicata, dovendosi introdurre termini correttivi non provenienti dalle interazioni fondamentali.
In un sistema non inerziale può quindi essere difficoltoso capire comprendere come agiscono le forze. 

\end{document}