\documentclass[class=book, crop=false, oneside, 12pt]{standalone}
\usepackage{standalone}
\usepackage{amsmath}
\usepackage{../../style}
% \graphicspath{{./assets/images/}}

% arara: pdflatex: { synctex: yes, shell: yes }
% arara: latexmk: { clean: partial }
\begin{document}

\section{Capitolo 1 - Cinematica del punto}

\subsection{Introduzione}

La meccanica riguarda lo studio del moto di un corpo: essa spiega la
relazione che esiste tra le cause che generano il moto e le
caratteristiche di questo e la esprime con leggi quantitative.

Iniziamo lo studio del moto dal più semplice corpo, quello puntiforme,
detto \emph{punto materiale} o spesso anche \emph{particella}: si tratta
di un corpo privo di dimensioni ovvero che presenti dimensioni
trascurabili rispetto a quelle dello spazio in cui può muoversi o degli
altri corpi con cui può interagire.

L' introduzione di tale concetto rende innanzitutto più semplice la
trattazione di alcuni aspetti di certi problemi. Inoltre lo studio del
sistema punto materiale permette di definire nel modo più facile alcune
grandezze meccaniche fondamentali e di capirne il significato con
immediatezza, in assenza delle complicazioni che deriverebbero dalla
struttura estesa del corpo.

Un corpo esteso solo eccezionalmente si muove come un punto materiale,
spesso infatti può compiere contemporaneamente altri tipi di moto.

L'analisi completa del moto comprende:

\begin{itemize}
\item
  il collegamento del moto stesso alle interazioni del corpo con i corpi
  circostanti
\item
  la descrizione geometrica dell'evoluzione temporale del fenomeno di
  movimento.
\end{itemize}

Questa parte della meccanica, descrittiva del moto di un corpo,
indipendentemente dalle cause che lo determinano, viene detta
\emph{cinematica}, mentre il perché del moto viene studiato nella
\emph{dinamica}.

\paragraph{Il moto}
Il \emph{moto} di un punto materiale è determinato se è nota la sua posizione
in funzione del tempo in un determinato sistema di riferimento, ossia ad
esempio le sue coordinate \(x(t), y(t),z(t)\) in un sistema di
riferimento cartesiano. Può non essere l'unico modo, infatti ci sono
altri sistemi di riferimento, ad esempio quelli con coordinate polari.

\paragraph{La traiettoria}
La \emph{traiettoria} è il luogo dei punti occupati successivamente dal punto
in movimento e costituisce una curva continua nello spazio. Lo studio
delle variazioni di posizione lungo la traiettoria nel tempo porterà a
definire il concetto di \emph{velocità}, mentre lo studio delle
variazioni della velocità con il tempo introdurrà la grandezza
\emph{accelerazione}.

\paragraph{Le grandezze fondamentali}
Le \emph{grandezze fondamentali} in cinematica sono pertanto lo
\emph{spazio}, la \emph{velocità}, l'\emph{accelerazione} e il
\emph{tempo}; quest'ultimo, in sostanza, rappresenta la variabile
indipendente.

\paragraph{La quiete}
La \emph{quiete} è un particolare tipo di moto in cui le coordinate
restano costanti e quindi velocità e accelerazione sono nulle. Questo
stato non è assoluto infatti è sempre relativo ad un sistema di
riferimento dal quale si osserva il moto. Lo stesso punto materiale può
apparire diverso in sistemi di riferimento differenti. Parleremo più
avanti di \emph{moto relativo}, per descrivere le relazioni che esistono
tra le descrizioni di uno stesso moto visto da due sistemi di
riferimento in movimento l'uno rispetto all'altro.


\subsection{Moto rettilineo}
Il moto rettilineo si svolge lungo una retta sulla quale vengono fissati
arbitrariamente un' origine e un verso; il moto è descrivibile tramite
una sola coordinata \(x(t)\).

Le misure ottenute possono essere rappresentate in un sistema con due
assi cartesiani. Sull' asse delle ordinate riportiamo i valori di \(x\)
e su quello delle ascisse i corrispondenti valori del tempo: la figura
si chiama \emph{diagramma orario}. È necessario ovviamente introdurre
delle unità di riferimento nei due assi, ad esempio la lunghezza
corrispondente ad un intervallo di tempo di un secondo nelle ascisse e
quella relativa ad uno spostamento di un metro nelle ordinate (SI).

È importante ricordare che in una misura fisica reale ciascun dato è
affetto da errori e pertanto i punti che rappresentano le varie misure
non si dispongono esattamente su una retta, una parabola o su altri tipi
di curve. L'espressione \(x(t)\) è ottenibile solo tramite opportuni
metodi di ottimizzazione analitica.

(Esempi)

\subsection{Velocità nel moto rettilineo}
Introduciamo ora il concetto di velocità media e velocità istantanea nel
moto rettilineo. Se al tempo \(t = t_1\) il punto si trova nella posizione \(x = x_1\) e al
tempo \(t = t_2\) nella posizione \(x = x_2\) ,
\(\Delta x = x_2 - x_1\) rappresenta lo spazio percorso nell'intervallo
di tempo \(\Delta t = t_2 - t_1\).

Possiamo caratterizzare la velocità dello spostamento tramite la
velocità media.

\begin{equation}
v_m = \frac {\Delta x} {\Delta t} = \frac {x_2 - x_1} {t_2 - t_1}
\end{equation}

Questa grandezza fornisce un informazione complessiva ma che ci dà poche
informazioni su come sta avvenendo il moto.

Per individuare \(x(t)\) e le sue variazioni aumentiamo il numero di
misure suddividendo lo spazio \(\Delta x\) in numerosi piccoli
intervalli \((\Delta x)_1,...,(\Delta x)_n\), percorsi in numerosi
intervalli di tempo \(\Delta t\) \((\Delta t)_1,...,(\Delta t)_n\). Da
questi piccoli intervalli posso ottenere tutte le corrispondenti
velocità medie \(v_1,...,v_n\). Di solito queste velocità non sono
uguali tra loro (nel caso lo fossero si parla di Moto Rettilineo
Uniforme o M.R.U) e di solito nemmeno uguali a \(v_m\).

Il processo di suddivisione in spazi sempre più piccoli può essere
continuato fino ad avere intervalli infinitesimi. Posso definire la
velocità istantanea, ad un istante \(t\) , del punto in movimento come
il rapporto \(v = dx/dt\) , calcolato in quel determinato istante.
Questa operazione corrisponde all'operazione matematica del calcolo del
limite per \(\Delta t \rightarrow 0\) del rapporto incrementale
\(\Delta x / \Delta t\).

Pertanto la velocità di un punto nel moto rettilineo è data dalla
derivata dello spazio rispetto al tempo: \(v = \frac {dx} {dt}\); la
velocità istantanea, cioè, rappresenta la rapidità di variazione
temporale della posizione nell' istante \(t\) considerato.

Il segno della velocità fa riferimento al verso del moto sull'asse
\(x\): se \(v >0\) la coordinata cresce, mentre se \(v < 0\) il moto
avviene nel verso opposto. A sua volta la velocità può essere funzione
del tempo \(v(t)\).

In conclusione, se è nota, perché calcolata o misurata, la funzione
\(x(t)\) ovvero, come si dice, se è nota la legge oraria, si può
ottenere la velocità istantanea con l'operazione di derivazione

Per risolvere il problema inverso, trovare \(x(t)\) sapendo \(v(t)\), mi
basta svolgere l'operazione contraria: l'integrazione. Ottengo che
\(\Delta x = \int_{x_0}^{x} dx = \int_{t_0}^{t} v(t) dt\). In
conclusione ho che \(x(t) = x_0 + \int_{t_0}^t v(t) dt\) dove \(x_0\)
rappresenta la posizione iniziale del punto all'istante \(t_0\).

\(\Delta x\) rappresenta lo spostamento complessivo quindi se il punto
ritorna nella posizione iniziale potrebbe annullarsi.

A partire dalla velocità media: \(v_m = \frac {x-x_0} {t-t_0}\),
ricaviamo la relazione tra velocità media e velocità istantanea:
\(v_m = \frac {1} {t-t_0} \int_{t_0}^t v(t) dt\).

Nel caso particolare di \(v = costante\) ottengo che:
\(x(t) = x_0 + vt\). Infatti, nel moto rettilineo uniforme, lo spazio è
una funzione lineare del tempo.

\subsection{Accelerazione nel moto rettilineo}

Se la velocità cambia in un intervallo di tempo definiamo la grandezza
\emph{accelerazione media} come \(a_m = \Delta v / \Delta t\).
Similmente a come siamo passati da velocità media a velocità istantanea
definiamo l'\emph{accelerazione istantanea}:
\(a = \frac {dv} {dt} = \frac {d^2 x}{dt^2}\)

Se \(a=0\) la velocità è costante, se \(a>0\) sto accelerando mentre se
\(a<0\) sto decelerando. Però è sempre il segno algebrico della velocità
a dirmi il verso del moto, non quello dell'accelerazione.

A partire da \(a(t)\) posso ricavare \(v(t)\) sempre tramite
integrazione:

\begin{equation}
  \begin{aligned}
    dv &= a(t) dt\\
    \Delta v = \int_{v_0}^{v}dv &= \int_{t_0}^t a(t) dt\\
    v(t) &= v_0 + \int_{t_0}^{t} a(t) dt
  \end{aligned}
\end{equation}


Anche qui \(v_0\) rappresenta la velocità iniziale all'istante di tempo
\(t_0\) e \(\Delta v\) rappresenta la variazione complessiva
della velocità.

Se l'accelerazione è costante si parla di moto rettilineo uniformemente
accelerato, dove la dipendenza della velocità dal tempo è lineare:
\(v(t) = v_0 + a(t-t_0)\) (ovviamente se \(t_0 = 0\) ottengo
\(v(t) = v_0 + at\)).

Da questa formula posso ricavare \(x(t)\) con:

\begin{equation}
  \begin{aligned}
    x(t) &= x_0 + \int_{t_0}^{t} [v_0 + a(t-t_0)] dt\\
         &= x_0 + \int_{t_0}^{t}v_0 dt+ \int_{t_0}^t a (t-t_0) dt\\
         &= x_0 + v_0(t-t_0) + \frac{1}{2} a(t-t_0)^2\\
  \end{aligned}
\end{equation}

Da cui poi:
\begin{equation}
  x(t) = x_0 + v_0 t + \frac {1} {2} a t^2 \qquad \text{con \(t_0=0\)}
\end{equation}

In conclusione se ho \(x(t)\) derivando posso ottenere \(v(t)\) e
\(a(t)\) a condizione che siano note le condizioni iniziali (velocità e
posizione nel punto \(t_0\)).

Se sono a conoscenza di \(a(x)\) (accelerazione rispetto la posizione)
posso ricavare il valore della velocità in ogni posizione \(x\),
\(v(x)\). Infatti se ad un certo istante \(t\) il punto occupa una
determinata posizione \(x\), con un valore \(v\) della velocità e \(a\)
dell'accelerazione, queste si possono pensare come funzioni della
posizione oltre che del tempo e si può scrivere
\(v(t) = v[x(t)], a(t)=a[x (t)]\). Utilizzando la derivazione a catena
ed integrando ottengo l'equazione di Torricelli, questo risultato può
essere ricavato analogamente sotto l'ipotesi di accelerazione costante
mediante sostituzione nelle equazioni:
\(s = s_0 + v_0\Delta t + \frac {1}{2} a(\Delta t)^2\) e
\(v = v_0 + a\Delta t\).

Nel moto uniformemente accelerato (\(a\) = costante) si ha:
\(v^2 = v_0^2 +2a (x-x_0)\).


\subsection{\texorpdfstring{Moto verticale di un corpo
}{Moto verticale di un corpo }}

Trascurando l'attrito con l'aria (dato che utilizzando i punti materiali
trascuriamo le dimensioni), un corpo lasciato libero di cadere in
vicinanza della superficie terrestre si muove verso il basso con una
accelerazione costante che vale in modulo \(g = 9.8 \ ms^{-2}\), il
verso dipende dal sistema di riferimento. Si tratta quindi di un Moto
Rettilineo Uniformemente Accelerato (MRUA).

In questo caso posso effettuare tre casi di studio per analizzare il
fenomeno: punto materiale lasciato cadere, punto materiale lanciato
verso il suolo e punto materiale lanciato verso l'alto, li vediamo
separatamente.

\subsubsection{Punto lasciato cadere}

Se lascio cadere un corpo (in un sistema di riferimento con asse \(x\)
che punta in alto), esso sarà sottoposto ad una accelerazione di
\(a = -g = -9.8 ms^{-2}\), partirà da un altezza iniziale
\(h = x_0 = 0\) con velocità nulla (\(v_0 = 0\)) al tempo iniziale
\(t = t_0  = 0\).

Ci troviamo nel MRUA so che \(v(t) = -gt\) e dalla legge oraria ottengo
\(x = h + \frac {1} {2} (-g)t^2\).

Il punto deve arrivare al suolo (\(x=0\)), il tempo impiegato si trova
con \(t_c = \sqrt {\frac {2h} {g}}\).

Dato che la velocità continua ad aumentare potrebbe interessarci
\(v(x)\), la velocità in funzione della posizione. Grazie all'equazione
di Torricelli ottengo che il punto arriva al suolo con velocità
\(v_c = \sqrt {2gh}\)

\subsubsection{Punto lanciato verso il basso}

A differenza del caso precedente, qui ho una velocità iniziale (\(v_0\))
non nulla che va considerata. Quindi \(v(t) = -v_0 -gt\) e la legge
oraria è \(x = h - v_0t - \frac {1} {2 } gt^2\).

Trovo \(t(x)\) per trovare il tempo di caduta:
\(t(x) = \frac {-v_0 + \sqrt {v_0^2 + 2g (h-x)}} {g}\). Ponendo \(x=0\)
ottengo \(t_c = \frac {-v_0 + \sqrt {v_0^2 + 2g h}} {g}\).

La velocità al suolo (sempre dall'equazione di Torricelli)
\(v_c = \sqrt {v_0^2 + 2gh}\).

\subsubsection{Punto lanciato verso l'alto}

Il punto parte dal suolo \(x_0 = 0\) in \(t_0 = 0\) con velocità
\(v_ 0 =v_2 > 0\). Mentre il punto sale la sua velocità inizierà a
decrescere. Il punto si fermerà nell'istante \(t_M = \frac {v_2} g\) e
nella posizione \(x_M = x(t_M) = \frac {v_2^2} {2g}\).

Quando \(v = 0\) il punto si fermerà e a questo punto ci troveremo nel
primo caso (punto lasciato cadere). Il tempo di caduta dal punto \(x_M\)
è:
\(t_c = \sqrt \frac {2h}{g} = \sqrt \frac {2 x_M}{g} = \sqrt \frac {v_2^2}{g^2} = \frac {v_2} {g} = t_m\).
La durata complessiva del moto è \(2t_M\). Ora posso ricavare \(t(x)\)
dalla legge oraria e \(v(x)\) dall'equazione di Torricelli:

\begin{itemize}
\item
  Ricavo
  \(t(x) = \frac {v_2 \pm {\sqrt{v_2^2 -gx}}}{g} = t_M \pm \sqrt {t_M^2 - \frac {2x}{g}}\).
\item
  Ricavo \(v(x) = \pm \sqrt {v_2^2 -2gx}\).
\end{itemize}

Il doppio segno sta ad indicare che il punto passa due volte per la
stessa posizione (all'inizio e alla fine del moto).

\subsection{Moto Armonico}
\paragraph{}
Il moto armonico semplice lungo un asse rettilineo è un moto vario la cui legge oraria è definita dalla relazione:
\begin{equation}
  x(t) = A \sin (\omega t + \phi)
\end{equation}
dove \(A,\omega, \phi \), sono grandezze costanti: \(A\) è detta ampiezza del moto, \(\omega t + \phi \) è la fase del moto dove
\(\phi \) è detta fase iniziale e \(\omega \) pulsazione.

Dato che la funzione seno assume valori da \(+1\) a \(-1\) dalla formula ottengo che il punto può apparire in un segmento lungo \(2A\) con centro nell'origine.
All'inizio del moto (istante di tempo \(t = 0\)) il punto occupa la posizione \(x(0) = A \cdot \sin (\phi) \).

Come la funzione seno, il moto risulta essere periodico, infatti esso descrive delle oscillazioni.
Queste oscillazioni hanno ampiezza \(A\) e centro \(0\), sono tutte uguali tra loro e caratterizzate dalla durata, detta periodo \(T\) del moto armonico.
\paragraph{Definizione}
Il moto di un punto si dice periodico quando ad intervalli di tempo eguali il punto ripassa nella stessa posizione con la stessa velocità.

Per determinare il periodo \(T\) considero due tempi \(t,t'\) con \(t'-t = T\).
Ho che per definizione \(x(t) = x(t')\), quindi imponendo l'ugualianza e sostituendo ottengo \(T = \frac {2 \pi} {\omega}\) oppure \(\omega = \frac{2 \pi}{T}\).
Quindi \(\omega\) e \(T\) sono inversamente proporzionali.
Definisco frequenza \(v\) del moto il numero di oscillazioni al secondo:
\(v = \frac{1}{T} = \frac{\omega}{2 \pi}\).

Il periodo e quindi la frequenza, sono indipendenti dall'ampiezza del moto.
Una volta fissato il valore della pulsazione abbiamo una classe di moti 
che differiscono tra loro per i diversi valori dell' ampiezza e della fase iniziale, cioè, come vedremo, per le diverse condizioni iniziali.

La velocità di un punto che si muove con moto armonico si ottiene derivando \(x(t)\):
\begin{equation}
  v(t) = \frac{dx}{dt} = \omega A \cos (\omega t + \phi)
\end{equation}
Con un ulteriore derivazione si ottine l'accelerazione del punto:
\begin{equation}
  a(t) = \frac{dv}{dt} = \frac{d^2 x}{dt^2} = -\omega^2 A \sin (\omega t + \phi) = -\omega^2 x
\end{equation}

Guardando i grafici delle tre funzioni osserviamo che: 
la velocità assume il valore massimo nel centro di oscillazione: (assunto come origine 0) dove vale \(\omega A\) e i annulla agli estremi (\(x =A\) e \(x =-A\)) dove si inverte il senso del moto. 
L'accelerazione si annulla nel centro di oscillazione e assume il valore massimo in modulo (\(\omega^2 A\) agli estremi, dove si inverte la velocità; inoltre essa è sempre proporzionale ed opposta allo spostamento dal centro di oscillazione. 

Le tre funzioni mostrano lo stesso andamento temporale però risultano sfasate una rispetto all'altra.
La velocità è sfasata di \(\pi / 2\) rispetto allo spostamento (è in quadratura di fase) mentre l'accelerazione è sfasata di \(\pi\) sempre rispetto allo spostamento (è in opposizione di fase).

Le costanti \(A\) e \(\phi\) individuano le condizioni iniziali:
\begin{equation}
  x(0) = x_0 = A \sin \phi \ , \
  v(0) = v_0 = \omega A \cos \phi
\end{equation}
invece le formula inverse sono:
\begin{equation}
  \tan \phi = \frac{\omega x_0}{v_0} \ , \ 
  A^2 = x_0^2 + \frac {v_0^2}{\omega^2}
\end{equation}
Sostituendo la formula dell'accelerazione del moto armonico nell'equazione di Torricelli posso trovare \(v(x)\):
\begin{equation}
  v^2(x) = \phi^2 (A^2 - x^2) 
\end{equation}

Dalla legge oraria abbiamo ricavato che l' accelerazione è proporzionale allo spostamento, con segno negativo: \(a = - \omega^2 x\). Se invece si trova che in un certo moto l'accelerazione risulta proporzionale allo spostamento con costante di proporzionalità negativa si dimostra che quel moto è armonico semplice. 
In altre parole, la condizione necessaria e sufficiente perché un moto sia armonico è data dall'equazione.
\begin{equation}
  \frac{d^2 x}{dt^2} + \omega^2 x = 0
\end{equation}
detta equazione differenziale del moto armonico. Le funzioni seno e coseno, e le loro combinazioni lineari, sono tutte e sole le funzioni che soddisfano alla condizione nel campo reale. 

Le proprietà generali del moto armonico semplice restano eguali se invece della funzione seno utilizziamo la funzione coseno. 
Le due funzioni differiscono solo per un termine di sfasamento: \(\frac{\pi}{2}\): \(\sin(\omega t + \phi) = \cos(\omega t + \zeta)\) con \(\zeta = \phi - \frac{\pi}{2}\)
Essi rappresentano lo stesso moto solo che il primo è visto a partire dall'istante $t_0$ e il secondo dall'istante \(t_0 + \frac{T}{4}\).

In questo caso di studio abbiamo visto solo i moti, ma ci possono essere altri fenomeni fisici dove c'è una grandezza \(f\) per cui vale un equazione con la struttura di (7):
\begin{equation}
  \frac {d^f} { dz^2} + k^2 f = 0
\end{equation}
in cui la soluzione è sempre \(f(z) = A \sin(kz + \phi)\) cioè \(f\) descrive un oscillazione rispetto alla variabile \(z\), il cui periodo dipende da \(k\).

\subsection{Moto nel piano. Posizione e Velocità}

Nel caso che il moto sia vincolato a svolgersi su di un piano la traiettoria del punto P,
è in generale  una linea curva. Dobbiamo quindi analizzare la direzione e verso del moto oltre al valore numerico dello spostamento.
Grandezze con queste caratteristiche si chiamano \emph{vettori}.

Un vettore è un segmento orientato caratterizzato da un modulo, una direzione ed un verso.

Nel piano un punto è espresso in due coordinate. Queste possono fare riferimento al classico sistema di assi cartesiani ortogonali, 
\(x(t)\) e \(y(t)\) oppure, in termini di coordinate polari nel piano. 
Le relazioni tra queste due sono:
\begin{equation}
  x = r \cos \Theta \ , \ y = r \sin \Theta
\end{equation}
\begin{equation}
  r = \sqrt{x^2 + y^2 } \ , \ \tan \Theta = \frac{y}{x}
\end{equation}

La posizione del punto \(P\) può anche essere individuata per mezzo del \emph{raggio vettore}:
\begin{equation}
  \overrightarrow{r}(t) = OP = x(t) \overrightarrow{u}_x + y(t) \overrightarrow{u}_y
\end{equation}
dove \(\overrightarrow{u}_x\) e \(\overrightarrow{u}_y\) rappresentano i versori degli assi cartesiani che si considerano fissi nel tempo.
Se è nota la dipendenza dal tempo di \(\overrightarrow{r}\), cioè la funzione \(\overrightarrow{r}(t)\), è individuato il moto del punto \(P\).

La posizione del punto lungo la traiettoria può anche essere data da una coordinata curvilinea \(s\), misurata a partire da un'origine arbitraria. Il valore di \(s\) esprime la lunghezza della traiettoria e varia nel tempo durante il moto: \(\frac{ds}{dt}\) indica la variazione temporale della posizione lungo la traiettoria cioè la velocità istantanea del punto, come definita nel modo rettilineo. 
Se diamo la forma della traiettoria e la velocità con cui viene percorsa abbiamo fornito una descrizione completa del moto. 
Consideriamo due posizioni occupate dal punto \(P\) al tempo \(t\) e al tempo \(t + \Delta t\): esse sono individuate dai vettori \(\overrightarrow{r}(t)\) e \(\overrightarrow{r}(t + \Delta t) = \overrightarrow{r}(t) + \Delta \overrightarrow{r}\).
Si costruisce il rapporto incrementale \(\frac{\overrightarrow{r}(t+\Delta t) - \overrightarrow{r}(t)}{\Delta t} = \frac{\Delta \overrightarrow{r}}{\Delta t}\) e si definisce velocità vettoriale il limite per \(\Delta t \rightarrow 0\):
\begin{equation}
  \overrightarrow{v} = \frac{d \overrightarrow{r}}{dt}
\end{equation}
La velocità vettoriale è la derivata del raggio vettore rispetto al tempo. Al limite l'incremento del raggio vettore risulta in direzione tangente alla traiettoria nel punto \(P\) e in modulo e uguale allo spostamento infinitesimo \(ds\) lungo la traiettoria, per cui possiamo scrivere \(d \overrightarrow{r} = ds \overrightarrow{u}_T\) dove \(\overrightarrow{u}_t\) è il versore della tangente alla curva, variabile nel tempo man mano che iL punto avanza lungo la traiettoria.

In sostanza pensiamo il moto come una successione di spostamenti rettilinei infinitesimi con direzione variabile: la direzione istantanea del moto coincide con quella della tangente alla traiettoria nel punto occupato all'istante considerato.
\begin{equation}
  \overrightarrow{v} = \frac{ds}{dt} \overrightarrow{u}_T = v \overrightarrow{u}_T
\end{equation}
pertanto la velocità vettoriale \(\overrightarrow{v}\) individua in ogni istante con la sua direzione e verso direzione e il suo moto e con il suo modulo \(v = \frac{ds}{dt}\) la velocità istantanea con cui è percorsa la traiettoria. 

Bisogna fare attenzione a non confondere i concetti di raggio vettore e i suoi incrementi finiti da una parte e il percorso effettivo dall'altra:
un punto potrebbe percorrere un'orbita chiusa ritornando al punto di partenza e in tal caso il raggio vettore non cambia, ma il punto ha percorso una traiettoria finita (\(\Delta \overrightarrow{r} = 0, \Delta s \neq 0\)) con velocità vettoriale istantanea diversa da zero (similmente a quello che accadeva nel MRU, risulta nulla la velocità vettoriale media).

La traiettoria del moto e la definizione di velocità \(v \overrightarrow{u}_T\) non dipendono dal sistema di rifermento (invarianza delle relazioni vettoriali rispetto alla scelta del sistema di riferimento).
Tuttavia un vettore che si esprime esplicitamente tramite le sue componenti cambia perciò queste dipendono dal sistema di riferimento.

\subsection{Accelerazione nel moto piano}

L'accelerazione nel moto piano deve esprimere le variazioni della velocità sia come \emph{modulo} che \emph{direzione} e quindi ci aspettiamo che abbia due componenti, 
una legata alla variazione del modulo della velocità e la seconda al cambiamento di direzione del moto. 
Nel moto rettilineo, dove la velocità mantiene sempre la stessa direzione, l'accelerazione è espressa da un solo termine. 

Anche nel moto piano l'accelerazione si definisce come derivata della velocità rispetto al tempo (ed è una grandezza vettoriale): 
\begin{equation}
  \overrightarrow{a} = \frac{d \overrightarrow{v}}{dt} = \frac{d^2 \overrightarrow{r}}{dt^2}
\end{equation}
Derivando le componenti del vettore:
\begin{equation}
  \overrightarrow{a}=\frac{d}{d t}\left(v \overrightarrow{u}_{\mathrm{T}}\right)=\frac{d v}{d t} \overrightarrow{u}_{\mathrm{T}}+v \frac{d \overrightarrow{u}_{\mathrm{T}}}{d t}=\frac{d v}{d t} \overrightarrow{u}_{\mathrm{T}}+v \frac{d \phi}{d t} \overrightarrow{u}_{\mathrm{N}}
\end{equation}
La prima componente, parallela alla velocità, esprime la variazione del modulo della velocità; il secondo termine, dipendente dalla variazione di direzione della velocità, è ortogonale a questa:
\(\overrightarrow{u}_N\) è un vettore ortogonale a \(\overrightarrow{u}_T\) diretto verso la concavità della traiettoria, e \(d \phi/dt \) dice quanto rapidamente cambia la direzione di \(\overrightarrow{u}_T\)  e quindi di \(\overrightarrow{u}_N\). 

Le due componenti dell'accelerazione si chiamano accelerazione tangenziale e accelerazione normale o centripeta. 
In un moto curvilineo vario entrambe le componenti sono diverse da zero; se però il moto curvilineo è uniforme è nulla \(\overrightarrow{a}_T\); se invece il moto è rettilineo vario è nulla \(\overrightarrow{a}_N\) e solo nel moto rettilineo uniforme \(\overrightarrow{a}_N\) = \(\overrightarrow{a}_T\) = \(\overrightarrow{0}\).
In altre parole con \(a_T \neq 0\) il moto è sempre vario, con \(a_N \neq 0\) è  sempre curvilineo. 

\end{document}