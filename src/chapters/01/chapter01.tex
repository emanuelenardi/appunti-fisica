\documentclass[class=book, crop=false, oneside, 12pt]{standalone}
\usepackage{standalone}
\usepackage{amsmath}
\usepackage{../../style}
% \graphicspath{{./assets/images/}}

% arara: pdflatex: { synctex: yes, shell: yes }
% arara: latexmk: { clean: partial }
\begin{document}

\chapter{Cinematica del punto}

\section{Introduzione}
La meccanica riguarda lo studio del moto di un corpo: spiega la relazione che esiste tra le cause che generano il moto e le sue caratteristiche esprimendole con leggi \(\overrightarrow{\mathbf{r}}(t)\) quantitative.
  \subsection{Punto materiale}
  Si definisce punto materiale un corpo puntiforme privo di dimensioni.
  \`E assimilabile a un corpo di dimensioni trascurabili rispetto a quelle dello spazio in cui si muove o degli altri corpi con cui interagisce.
  L'utilizzo di un punto materiale rende l'analisi moto pi\`u semplice in quanto rende trascurabili le propriet\`a fisiche del corpo, come la sua forma o dimensione, ininfluenti sul moto stesso.
    Inoltre, a differenza di un corpo esteso non pu\`o compiere rotazioni o vibrazioni.
  \subsection{Cinematica}
  Un analisi completa del moto comprende:
    \begin{itemize}
      \item Il collegamento del moto alle interazioni del corpo in esame con i corpi circostanti.
      \item La descrizione geometrica dell'evoluzione temporale del fenomeno di movimento.
    \end{itemize}
  Si dice \emph{cinematica} lo studio descrittivo del moto di un corpo indipendente dalle cause che lo determinano.
    \subsubsection{Concetti fondamentali}
      \paragraph{Moto}
      Il moto di un punto materiale \`e determinato se nota la sua posizione in funzione del tempo in un determinato sistema di riferimento, sia questo cartesiano (\(x(t), y(t), z(t)\)) o polare.
      \paragraph{Traiettoria}
      La traiettoria \`e il luogo dei punti occupati successivamente dal punto in movimento.
      Costituisce pertanto una curva continua nello spazio.
      \paragraph{Grandezze fondamentali}
      \begin{itemize}
        \item Spazio: la posizione del corpo.
        \item Velocit\`a: la variazione di posizione lungo la traiettoria.
        \item Accelerazione: variazioni della velocità nel tempo.
      \end{itemize}
\section{Quiete}
La \emph{quiete} è un particolare tipo di moto in cui le coordinate restano costanti e quindi velocità e accelerazione sono nulle.
Si noti come questo stato non \`e assoluto ma relativo al sistema di riferimento da cui si osserva il moto.
Lo stesso \emph{punto materiale} può apparire diverso in sistemi di riferimento differenti.
\section{Moto rettilineo}
Il moto rettilineo si svolge lungo una retta sulla quale vengono fissati arbitrariamente un'origine e un verso ed è pertanto descrivibile tramite una sola coordinata \(x(t)\).

Le misure ottenute possono essere rappresentate in un sistema con due assi cartesiani detto \emph{diagramma orario}: sull'asse delle ordinate riportiamo i valori di \(x\) (metri) e su quello delle ascisse i corrispondenti valori del tempo (secondi).
(Esempi)
  \subsection{Velocità}
    \subsubsection{Velocit\`a media}
    \begin{equation*}
      v_m = \frac {\Delta x} {\Delta t} = \frac {x_2 - x_1} {t_2 - t_1} \udm{\frac {m} {s}}
    \end{equation*}
    \subsubsection{Velocit\`a istantanea}
    Per sopperire alla mancanza di informazioni della velocit\`a media, in modo da individuare \(x(t)\) e le sue variazioni si suddivide \(\Delta t\) in numerosi piccoli intervalli \((\Delta x)_i\) percorsi in altrettanti piccoli intervalli $\Delta t\ (\Delta t)_i$.
    Si noti come le corrispondenti velocit\`a medie $v_i=\frac{(\Delta x)_i}{(\Delta t)_i}$ sono diverse tra di loro e da $v_m$, in quanto in un generico moto rettilineo la velocit\`a non \`e costante nel tempo.
    Suddividendo $\Delta x$ in un numero elevatissimo di intervalli $dx$ infinitamente piccoli si pu\`o definire la velocit\`a istantanea in un istante $t$ del punto in movimento come:
    \begin{equation*}
      v=\lim\limits_{\Delta t\rightarrow 0}\dfrac{\Delta x}{\Delta t}=\dfrac{dx}{dt}
    \end{equation*}
    La velocit\`a istantanea rappresenta pertanto la rapidit\`a di variazione temporale della posizione nell'istante $t$ considerato, pu\`o pertanto essere espressa in funzione del tempo $v(t)$.
    Il segno fa riferimento al verso del moto su $x$.

In conclusione, se è nota, perché calcolata o misurata, la funzione
\(x(t)\) ovvero, come si dice, se è nota la legge oraria, si può
ottenere la velocità istantanea con l'operazione di derivazione

Per risolvere il problema inverso, trovare \(x(t)\) sapendo \(v(t)\), mi
basta svolgere l'operazione contraria: l'integrazione. Ottengo che
\(\Delta x = \int_{x_0}^{x} dx = \int_{t_0}^{t} v(t) dt\). In
conclusione ho che \(x(t) = x_0 + \int_{t_0}^t v(t) dt\) dove \(x_0\)
rappresenta la posizione iniziale del punto all'istante \(t_0\).

\(\Delta x\) rappresenta lo spostamento complessivo quindi se il punto
ritorna nella posizione iniziale potrebbe annullarsi.

A partire dalla velocità media: \(v_m = \frac {x-x_0} {t-t_0}\),
ricaviamo la relazione tra velocità media e velocità istantanea:
\(v_m = \frac {1} {t-t_0} \int_{t_0}^t v(t) dt\).

Nel caso particolare di \(v = costante\) ottengo che:
\(x(t) = x_0 + vt\). Infatti, nel moto rettilineo uniforme, lo spazio è
una funzione lineare del tempo.

\section{Accelerazione nel moto rettilineo}

Se la velocità cambia in un intervallo di tempo definiamo la grandezza
\emph{accelerazione media} come \(a_m = \frac{\Delta v} {\Delta t} \udm{\frac{m}{s^2}}\).
Similmente a come siamo passati da velocità media a velocità istantanea
definiamo l'\emph{accelerazione istantanea}:
\(a = \frac {dv} {dt} = \frac {d^2 x}{dt^2} \udm{\frac{m}{s^2}}\)

Se \(a=0\) la velocità è costante, se \(a>0\) sto accelerando mentre se
\(a<0\) sto decelerando. Però è sempre il segno algebrico della velocità
a dirmi il verso del moto, non quello dell'accelerazione.

A partire da \(a(t)\) posso ricavare \(v(t)\) sempre tramite
integrazione:

\begin{equation}
  \begin{aligned}
    dv &= a(t) dt\\
    \Delta v = \int_{v_0}^{v}dv &= \int_{t_0}^t a(t) dt\\
    v(t) &= v_0 + \int_{t_0}^{t} a(t) dt
  \end{aligned}
\end{equation}


Anche qui \(v_0\) rappresenta la velocità iniziale all'istante di tempo
\(t_0\) e \(\Delta v\) rappresenta la variazione complessiva
della velocità.

Se l'accelerazione è costante si parla di moto rettilineo uniformemente
accelerato, dove la dipendenza della velocità dal tempo è lineare:
\(v(t) = v_0 + a(t-t_0)\) (ovviamente se \(t_0 = 0\) ottengo
\(v(t) = v_0 + at\)).

Da questa formula posso ricavare \(x(t)\) con:

\begin{equation}
  \begin{aligned}
    x(t) &= x_0 + \int_{t_0}^{t} [v_0 + a(t-t_0)] dt\\
         &= x_0 + \int_{t_0}^{t}v_0 dt+ \int_{t_0}^t a (t-t_0) dt\\
         &= x_0 + v_0(t-t_0) + \frac{1}{2} a(t-t_0)^2\\
  \end{aligned}
\end{equation}

Da cui poi:
\begin{equation}
  x(t) = x_0 + v_0 t + \frac {1} {2} a t^2 \qquad \text{con \(t_0=0\)}
\end{equation}

In conclusione se ho \(x(t)\) derivando posso ottenere \(v(t)\) e
\(a(t)\) a condizione che siano note le condizioni iniziali (velocità e
posizione nel punto \(t_0\)).

Se sono a conoscenza di \(a(x)\) (accelerazione rispetto la posizione)
posso ricavare il valore della velocità in ogni posizione \(x\),
\(v(x)\). Infatti se ad un certo istante \(t\) il punto occupa una
determinata posizione \(x\), con un valore \(v\) della velocità e \(a\)
dell'accelerazione, queste si possono pensare come funzioni della
posizione oltre che del tempo e si può scrivere
\(v(t) = v[x(t)], a(t)=a[x (t)]\). Utilizzando la derivazione a catena
ed integrando ottengo l'equazione di Torricelli; questo risultato può
essere ricavato analogamente sotto l'ipotesi di accelerazione costante
mediante sostituzione nelle equazioni:
\(s = s_0 + v_0\Delta t + \frac {1}{2} a(\Delta t)^2\) e
\(v = v_0 + a\Delta t\).

Nel moto uniformemente accelerato (\(a\) = costante) si ha:
\begin{equation}
  v^2 = v_0^2 +2a (x-x_0)
\end{equation}


\section{Moto verticale di un corpo }

Trascurando l'attrito con l'aria (dato che utilizzando i punti materiali
trascuriamo le dimensioni), un corpo lasciato libero di cadere in
vicinanza della superficie terrestre si muove verso il basso con una
accelerazione costante che vale in modulo \(g = 9.8 \ ms^{-2}\), il
verso dipende dal sistema di riferimento. Si tratta quindi di un Moto
Rettilineo Uniformemente Accelerato (MRUA).

In questo caso posso effettuare tre casi di studio per analizzare il
fenomeno: \emph{punto materiale} lasciato cadere, \emph{punto materiale} lanciato
verso il suolo e \emph{punto materiale} lanciato verso l'alto, li vediamo
separatamente.

\subsection{Punto lasciato cadere}

Se lascio cadere un corpo (in un sistema di riferimento con asse \(x\)
che punta in alto), esso sarà sottoposto ad una accelerazione di
\(a = -g = -9.8 ms^{-2}\), partirà da un altezza iniziale
\(h = x_0 = 0\) con velocità nulla (\(v_0 = 0\)) al tempo iniziale
\(t = t_0  = 0\).

Ci troviamo nel MRUA so che \(v(t) = -gt\) e dalla legge oraria ottengo
\(x = h + \frac {1} {2} (-g)t^2\).

Il punto deve arrivare al suolo (\(x=0\)), il tempo impiegato si trova
con \(t_c = \sqrt {\frac {2h} {g}}\).

Dato che la velocità continua ad aumentare potrebbe interessarci
\(v(x)\), la velocità in funzione della posizione. Grazie all'equazione
di Torricelli ottengo che il punto arriva al suolo con velocità
\(v_c = \sqrt {2gh}\)

\subsection{Punto lanciato verso il basso}

A differenza del caso precedente, qui ho una velocità iniziale (\(v_0\))
non nulla che va considerata. Quindi \(v(t) = -v_0 -gt\) e la legge
oraria è
\begin{equation}
  x = h - v_0t - \frac {1} {2 } gt^2
\end{equation}

Trovo \(t(x)\) per trovare il tempo di caduta:
\(t(x) = \frac {-v_0 + \sqrt {v_0^2 + 2g (h-x)}} {g}\). Ponendo \(x=0\)
ottengo \(t_c = \frac {-v_0 + \sqrt {v_0^2 + 2g h}} {g}\).

La velocità al suolo (sempre dall'equazione di Torricelli)
\(v_c = \sqrt {v_0^2 + 2gh}\).

\subsection{Punto lanciato verso l'alto}

Il punto parte dal suolo \(x_0 = 0\) in \(t_0 = 0\) con velocità
\(v_ 0 =v_2 > 0\). Mentre il punto sale la sua velocità inizierà a
decrescere. Il punto si fermerà nell'istante \(t_M = \frac {v_2} g\) e
nella posizione \(x_M = x(t_M) = \frac {v_2^2} {2g}\).

Quando \(v = 0\) il punto si fermerà e a questo punto ci troveremo nel
primo caso (punto lasciato cadere). Il tempo di caduta dal punto \(x_M\)
è:
\(t_c = \sqrt \frac {2h}{g} = \sqrt \frac {2 x_M}{g} = \sqrt \frac {v_2^2}{g^2} = \frac {v_2} {g} = t_m\).
La durata complessiva del moto è \(2t_M\). Ora posso ricavare \(t(x)\)
dalla legge oraria e \(v(x)\) dall'equazione di Torricelli:

\begin{itemize}
\item
  Ricavo
  \(t(x) = \frac {v_2 \pm {\sqrt{v_2^2 -gx}}}{g} = t_M \pm \sqrt {t_M^2 - \frac {2x}{g}}\).
\item
  Ricavo \(v(x) = \pm \sqrt {v_2^2 -2gx}\).
\end{itemize}

Il doppio segno sta ad indicare che il punto passa due volte per la
stessa posizione (all'inizio e alla fine del moto).

\section{Moto Armonico}
\paragraph{}
Il moto armonico semplice lungo un asse rettilineo è un moto vario la cui legge oraria è definita dalla relazione:
\begin{equation}
  x(t) = A \sin (\omega t + \phi)
\end{equation}
dove \(A,\omega, \phi \), sono grandezze costanti: \(A\) è detta ampiezza del moto, \(\omega t + \phi \) è la fase del moto dove
\(\phi \) è detta fase iniziale e \(\omega \) pulsazione.

Dato che la funzione seno assume valori da \(+1\) a \(-1\) dalla formula ottengo che il punto può apparire in un segmento lungo \(2A\) con centro nell'origine.
All'inizio del moto (istante di tempo \(t = 0\)) il punto occupa la posizione \(x(0) = A \cdot \sin (\phi) \).

Come la funzione seno, il moto risulta essere periodico, infatti esso descrive delle oscillazioni.
Queste oscillazioni hanno ampiezza \(A\) e centro \(0\), sono tutte uguali tra loro e caratterizzate dalla durata, detta periodo \(T\) del moto armonico.
\paragraph{Definizione}
Il moto di un punto si dice periodico quando ad intervalli di tempo eguali il punto ripassa nella stessa posizione con la stessa velocità.

Per determinare il periodo \(T\) considero due tempi \(t,t'\) con \(t'-t = T\).
Ho che per definizione \(x(t) = x(t')\), quindi imponendo l'ugualianza e sostituendo ottengo \(T = \frac {2 \pi} {\omega}\) oppure \(\omega = \frac{2 \pi}{T}\).
Quindi \(\omega\) e \(T\) sono inversamente proporzionali.
Definisco frequenza \(v\) del moto il numero di oscillazioni al secondo:
\(v = \frac{1}{T} = \frac{\omega}{2 \pi}\).

Il periodo e quindi la frequenza, sono indipendenti dall'ampiezza del moto.
Una volta fissato il valore della pulsazione abbiamo una classe di moti
che differiscono tra loro per i diversi valori dell'ampiezza e della fase iniziale, cioè, come vedremo, per le diverse condizioni iniziali.

La velocità di un punto che si muove con moto armonico si ottiene derivando \(x(t)\):
\begin{equation}
  v(t) = \frac{dx}{dt} = \omega A \cos (\omega t + \phi)
\end{equation}
Con un'ulteriore derivazione si ottiene l'accelerazione del punto:
\begin{equation}
  a(t) = \frac{dv}{dt} = \frac{d^2 x}{dt^2} = -\omega^2 A \sin (\omega t + \phi) = -\omega^2 x
\end{equation}

Guardando i grafici delle tre funzioni osserviamo che:
la velocità assume il valore massimo nel centro di oscillazione: (assunto come origine 0) dove vale \(\omega A\) e i annulla agli estremi (\(x =A\) e \(x =-A\)) dove si inverte il senso del moto.
L'accelerazione si annulla nel centro di oscillazione e assume il valore massimo in modulo \(\omega^2 A\) agli estremi, dove si inverte la velocità; inoltre essa è sempre proporzionale ed opposta allo spostamento dal centro di oscillazione.

Le tre funzioni mostrano lo stesso andamento temporale però risultano sfasate una rispetto all'altra.
La velocità è sfasata di \(\pi / 2\) rispetto allo spostamento (è in quadratura di fase) mentre l'accelerazione è sfasata di \(\pi\) sempre rispetto allo spostamento (è in opposizione di fase).

Le costanti \(A\) e \(\phi\) individuano le condizioni iniziali:
\begin{equation}
  x(0) = x_0 = A \sin \phi \ , \
  v(0) = v_0 = \omega A \cos \phi
\end{equation}
invece le formula inverse sono:
\begin{equation}
  \tan \phi = \frac{\omega x_0}{v_0} \ , \
  A^2 = x_0^2 + \frac {v_0^2}{\omega^2}
\end{equation}
Sostituendo la formula dell'accelerazione del moto armonico nell'equazione di Torricelli posso trovare \(v(x)\):
\begin{equation}
  v^2(x) = \phi^2 (A^2 - x^2)
\end{equation}

Dalla legge oraria abbiamo ricavato che l'accelerazione è proporzionale allo spostamento, con segno negativo: \(a = - \omega^2 x\). Se invece si trova che in un certo moto l'accelerazione risulta proporzionale allo spostamento con costante di proporzionalità negativa si dimostra che quel moto è armonico semplice.
In altre parole, la condizione necessaria e sufficiente perché un moto sia armonico è data dall'equazione
\begin{equation}
  \frac{d^2 x}{dt^2} + \omega^2 x = 0
\end{equation}
detta equazione differenziale del moto armonico. Le funzioni seno e coseno, e le loro combinazioni lineari, sono tutte e sole le funzioni che soddisfano alla condizione nel campo reale.

Le proprietà generali del moto armonico semplice restano eguali se invece della funzione seno utilizziamo la funzione coseno.\\
Le due funzioni differiscono solo per un termine di sfasamento: \(\frac{\pi}{2}\): \(\sin(\omega t + \phi) = \cos(\omega t + \zeta)\) con \(\zeta = \phi - \frac{\pi}{2}\)
Essi rappresentano lo stesso moto solo che il primo è visto a partire dall'istante $t_0$ e il secondo dall'istante \(t_0 + \frac{T}{4}\).

In questo caso di studio abbiamo visto solo i moti, ma ci possono essere altri fenomeni fisici dove c'è una grandezza \(f\) per cui vale un equazione con la struttura di (7):
\begin{equation}
  \frac {d^f} { dz^2} + k^2 f = 0
\end{equation}
in cui la soluzione è sempre \(f(z) = A \sin(kz + \phi)\) cioè \(f\) descrive un oscillazione rispetto alla variabile \(z\), il cui periodo dipende da \(k\).

\section{Moto nel piano. Posizione e Velocità}

Nel caso che il moto sia vincolato a svolgersi su di un piano la traiettoria del punto P,
è in generale  una linea curva. Dobbiamo quindi analizzare la direzione e verso del moto oltre al valore numerico dello spostamento.
Grandezze con queste caratteristiche si chiamano \emph{vettori}.

Un vettore è un segmento orientato caratterizzato da un modulo, una direzione ed un verso.

Nel piano un punto è espresso in due coordinate. Queste possono fare riferimento al classico sistema di assi cartesiani ortogonali,
\(x(t)\) e \(y(t)\) oppure, in termini di coordinate polari nel piano.
Le relazioni tra queste due sono:
\begin{equation}
  x = r \cos \theta \ , \ y = r \sin \theta
\end{equation}
\begin{equation}
  r = \sqrt{x^2 + y^2 } \ , \ \tan \theta = \frac{y}{x}
\end{equation}

La posizione del punto \(P\) può anche essere individuata per mezzo del \emph{raggio vettore} (segmento che va dall'origine \(O\) fino al punto \(P\)):
\begin{equation}
  \overrightarrow{r}(t) = OP = x(t) \overrightarrow{u}_x + y(t) \overrightarrow{u}_y
\end{equation}
dove \(\overrightarrow{u}_x\) e \(\overrightarrow{u}_y\) rappresentano i versori degli assi cartesiani che si considerano fissi nel tempo.
Se è nota la dipendenza dal tempo di \(\overrightarrow{r}\), cioè la funzione \(\overrightarrow{r}(t)\), è individuato il moto del punto \(P\).

La posizione del punto lungo la traiettoria può anche essere data da una coordinata curvilinea \(s\) (direzione tangente punto per punto), misurata a partire da un'origine arbitraria. Il valore di \(s\) esprime la lunghezza della traiettoria e varia nel tempo durante il moto: \(\frac{ds}{dt}\) indica la variazione temporale della posizione lungo la traiettoria cioè la velocità istantanea del punto, come definita nel modo rettilineo.
Se diamo la forma della traiettoria e la velocità con cui viene percorsa abbiamo fornito una descrizione completa del moto.

Consideriamo due posizioni occupate dal punto \(P\) al tempo \(t\) e al tempo \(t + \Delta t\): esse sono individuate dai vettori \(\overrightarrow{r}(t)\) e \(\overrightarrow{r}(t + \Delta t) = \overrightarrow{r}(t) + \Delta \overrightarrow{r}\).
Si costruisce il rapporto incrementale \(\frac{\overrightarrow{r}(t+\Delta t) - \overrightarrow{r}(t)}{\Delta t} = \frac{\Delta \overrightarrow{r}}{\Delta t}\) e si definisce velocità vettoriale il limite per \(\Delta t \rightarrow 0\):
\begin{equation}
  \overrightarrow{v} = \frac{d \overrightarrow{r}}{dt}
\end{equation}
La velocità vettoriale è la derivata del raggio vettore rispetto al tempo.

Al limite l'incremento del raggio vettore risulta in direzione tangente alla traiettoria nel punto \(P\) e in modulo e uguale allo spostamento infinitesimo \(ds\) lungo la traiettoria,
per cui possiamo scrivere \(d \overrightarrow{r} = ds \overrightarrow{u}_T\) dove \(\overrightarrow{u}_t\) è il versore della tangente alla curva, variabile nel tempo man mano che il punto avanza lungo la traiettoria.
Questo non vale se non si è all'infinitesimo, lo si vede chiaramente con spostamenti \(\Delta s\) più grandi.

In sostanza pensiamo il moto come una successione di spostamenti rettilinei infinitesimi con direzione variabile: la direzione istantanea del moto coincide con quella della tangente alla traiettoria nel punto occupato all'istante considerato.
\begin{equation}
  \overrightarrow{v} = \frac{ds}{dt} \overrightarrow{u}_T = v \overrightarrow{u}_T
\end{equation}
Pertanto la velocità vettoriale \(\overrightarrow{v}\) individua in ogni istante con la sua direzione e verso direzione e il suo moto e con il suo modulo \(v = \frac{ds}{dt}\) la velocità istantanea con cui è percorsa la traiettoria.

Bisogna fare attenzione a non confondere i concetti di raggio vettore e i suoi incrementi finiti da una parte e il percorso effettivo dall'altra:
un punto potrebbe percorrere un'orbita chiusa ritornando al punto di partenza e in tal caso il raggio vettore non cambia, ma il punto ha percorso una traiettoria finita (\(\Delta \overrightarrow{r} = 0, \Delta s \neq 0\)) con velocità vettoriale istantanea diversa da zero (similmente a quello che accadeva nel MRU, risulta nulla la velocità vettoriale media).

La traiettoria del moto e la definizione di velocità \(v \overrightarrow{u}_T\) non dipendono dal sistema di riferimento (invarianza delle relazioni vettoriali rispetto alla scelta del sistema di riferimento).
Tuttavia un vettore che si esprime esplicitamente tramite le sue componenti cambia perciò queste dipendono dal sistema di riferimento.

\section{Accelerazione nel moto piano}

L'accelerazione nel moto piano deve esprimere le variazioni della velocità sia come \emph{modulo} che \emph{direzione} e quindi ci aspettiamo che abbia due componenti,
una legata alla variazione del modulo della velocità e la seconda al cambiamento di direzione del moto.
Nel moto rettilineo, dove la velocità mantiene sempre la stessa direzione, l'accelerazione è espressa da un solo termine.

Anche nel moto piano l'accelerazione si definisce come derivata della velocità rispetto al tempo (ed è una grandezza vettoriale):
\begin{equation}
  \overrightarrow{a} = \frac{d \overrightarrow{v}}{dt} = \frac{d^2 \overrightarrow{r}}{dt^2}
\end{equation}
Derivando le componenti del vettore:
\begin{equation}
  \overrightarrow{a}=\frac{d}{d t}\left(v \overrightarrow{u}_{\mathrm{T}}\right)=\frac{d v}{d t} \overrightarrow{u}_{\mathrm{T}}+v \frac{d \overrightarrow{u}_{\mathrm{T}}}{d t}=\frac{d v}{d t} \overrightarrow{u}_{\mathrm{T}}+v \frac{d \phi}{d t} \overrightarrow{u}_{\mathrm{N}}
\end{equation}
La prima componente, parallela alla velocità, esprime la variazione del modulo della velocità; il secondo termine, dipendente dalla variazione di direzione della velocità, è ortogonale a questa:
\(\overrightarrow{u}_N\) è un vettore ortogonale a \(\overrightarrow{u}_T\) diretto verso la concavità della traiettoria, e \(d \phi/dt \) dice quanto rapidamente cambia la direzione di \(\overrightarrow{u}_T\)  e quindi di \(\overrightarrow{u}_N\).

Le due componenti dell'accelerazione si chiamano accelerazione tangenziale e accelerazione normale o centripeta.
In un moto curvilineo vario entrambe le componenti sono diverse da zero; se però il moto curvilineo è uniforme è nulla \(\overrightarrow{a}_T\); se invece il moto è rettilineo vario è nulla \(\overrightarrow{a}_N\) e solo nel moto rettilineo uniforme \(\overrightarrow{a}_N\) = \(\overrightarrow{a}_T\) = \(\overrightarrow{0}\).
In altre parole con \(a_T \neq 0\) il moto è sempre vario, con \(a_N \neq 0\) è  sempre curvilineo.

\section{Moto circolare}

Si chiama moto circolare un moto piano la cui traiettoria è rappresentata da una circonferenza. Considerando che la velocità varia continuamente direzione l'accelerazione centripeta è sempre diversa da zero.
Nel moto circolare uniforme la velocità è costante in modulo e l'accelerazione tangente è nulla per cui \(\overrightarrow{a} = \overrightarrow{a}_N\);
se invece il modulo della velocità cambia nel tempo il moto circolare non è uniforme e \(\overrightarrow{a}_T\) è diversa da zero (quindi c'è anche una forza tangenziale).

Il moto circolare può essere descritto facendo riferimento allo spazio percorso sulla circonferenza \(s(t)\) oppure
utilizzando l'angolo \(\theta (t)\) sotteso dall'arco \(s(t)\), con \(\theta(t) = s(t)/R\).
Se assumiamo come variabile l'angolo ci inseriamo in un sistema di coordinate polari con centro \(0\) in cui il moto avviene \(r(t) = R = costante\) e \(\theta(t)\) variabile.

Siamo naturalmente interessati alle variazioni dell'angolo nel tempo e pertanto definiamo la \emph{velocità angolare} come la derivata dell'angolo rispetto al tempo:
\begin{equation}
  \omega = \frac{d \theta}{dt} = \frac{1}{R} \frac {ds} {dt} = \frac{v}{R} \udm{\frac{rad}{s}}
\end{equation}

Il moto circolare più semplice è quello uniforme (costanza modulo della velocità): \(v\) e \(\omega\) sono costanti e le leggi orarie sono:
\begin{equation}
  s(t) = s_0 + vt
\end{equation}
\begin{equation}
  \theta(t) = \theta + \omega t
\end{equation}
Il moto circolare uniforme ha accelerazione costante ortogonale alla traiettoria
\begin{equation}
  a = a_N = \frac{v^2}{R} = \omega^2 R
\end{equation}

È un moto periodico con periodo \(T = \frac{2 \pi R}{v} = \frac{2 \pi}{\omega}\), corrispondente al tempo necessario per compiere un giro completo.

I moti proiettati sugli assi cartesiani sono:
\begin{equation}
  x = R \cos \theta = R \cos (\omega t + \omega_0) \
  y = R \sin \theta = R \sin (\omega t + \omega_0)
\end{equation}

Nel caso del moto circolare non uniforme oltre all'accelerazione centripeta, che è variabile perché la velocità varia anche in modulo, dobbiamo considerare anche l'accelerazione tangenziale \(a_T = \frac{dv}{dt} \).

\section{Moto parabolico dei corpi}

Il moto (nel vuoto) di un punto \(P\) lanciato dall'origine \(0\) con velocità iniziale \(v_0\) formante un angolo \(\alpha\) con l'asse delle ascisse (asse orizzontale).\\
In particolare vogliamo calcolare la traiettoria, la massima altezza raggiunta e la posizione \(G\) in cui il punto ricade sull'asse x ovvero la gittata OG.

Il moto è caratterizzato da un accelerazione costante \(a = g = -g u_y\) e le condizioni iniziali sono \(r = O\) e \(v = v_0\) al tempo \(t = O\), istante di lancio.
Le velocità dei moti proiettati sugli assi sono \(v_x = v_0 \cos \alpha\) (costante) e \(v_y = v_0 \sin \alpha - g t\).
Quindi le leggi orarie dei moti proiettati sono
\begin{equation}
  x = v_0 \cos \alpha t \ , \ y = v_0 + \sin \alpha t - \frac{1}{2} g t^2
\end{equation}
Sull'asse \(x\) il moto è uniforme, sull'asse \(y\) uniformemente accelerato.

La traiettoria si trova esprimendo \(x(t)\) in funzione di \(x\) rispetto al tempo e sostituendo in \(y(t)\)
\begin{equation}
  y(x) = x \tan \alpha - \frac{g}{2 v_0^2 \cos^2 \alpha} x^2
\end{equation}
che è l'equazione di una parabola.
La direzione del moto, in funzione del tempo o della coordinata \(x\), può essere caratterizzata dall'angolo \(\phi\) che il vettore velocità forma con l'asse orizzontale:
\begin{equation}
  \tan \phi = \frac{v_y}{v_x} = \tan \alpha - \frac{g}{v_0 \cos \alpha} t = \tan \alpha - \frac{g}{v_0^2 \cos^2 \alpha} x
\end{equation}

Per calcolare la \emph{gittata}, imponiamo \(y(x) = 0\) e otteniamo due soluzioni: \(x=0\) e
\begin{equation}
  x_{G}=\frac{2 v_{0}^{2} \cos ^{2} \alpha \tan \alpha}{g}=\frac{2 v_{0}^{2} \cos \alpha \sin \alpha}{g}=2 x_{M}
\end{equation}
con \(x_M = v_0^2 \cos \alpha \sin \alpha / g\) coordinata dal punto di metto del segmento OG e quindi per la simmetria della parabola, ascissa del punto di massima altezza.
L'altezza massima raggiunta è pertanto
\begin{equation}
  y(x_M) = y_M = \frac{v_0^2 \sin^2 \alpha}{2g}
\end{equation}

Il tempo totale di volo \(t_G\) è pari al tempo impiegato a percorrere la traiettoria con velocità costante \(v_x = v_0 \cos \alpha : t_G = 2 x_M /v_0 \cos \alpha = 2v_0 \sin \alpha / g = 2 t_M\);
evidentemente \(t_G\) coincide con il tempo necessario per salire all'altezza \(y_M\) e ritornare al suolo.

Notiamo infine che nella posizione \(G\) la velocità è la stessa in modulo che alla partenza, ma è posta simmetricamente rispetto all'asse \(x\):
\begin{equation}
  v_x(t_G) = v_0 \cos \alpha \ , \ v_y (t_G) = -v_0 \sin \alpha \ , \ \tan \phi = - \tan \alpha
\end{equation}

Le caratteristiche geometriche del moto parabolico di un corpo vicino alla superficie terrestre si comprendono chiaramente nel sistema cartesiano adottato che è in definitiva il più naturale in questo problema in cui c'è una direzione di particolare importanza,
quella di \(g\), a 90° con una direzione di uso pratico molto comune, quella orizzontale.\\
Invece, per esempio, nella trattazione del moto circolare è certamente più semplice servirsi di un sistema di coordinate polari con centro nel centro di simmetria del sistema, mentre non hanno particolare significato gli assi \(x\) e \(y\).

\section{Riepilogo}

\begin{itemize}
  \item La descrizione del moto consiste nel determinare come varia la posizione del punto in funzione del tempo; se conosciamo la posizione ad un certo istante possiamo sapere come si sviluppa il moto se abbiamo la direzione istantanea del moto e la rapidità con cui avviene lo spostamento: ciò porta al concetto di velocità e possiamo concepire il moto come una successione di spostamenti infinitesimi \(d \overrightarrow{r}= \overrightarrow{v} dt\).
  \item La velocità a sua volta può cambiare durante il moto e le sue variazioni sono espresse dall'accelerazione : \(d \overrightarrow{v} = \overrightarrow{\alpha} d t\).
  \item La traiettoria è il luogo dei punti occupati in funzione del tempo dalla punta del vettore \(\overrightarrow{r}(t)\): questo dà la posizione del punto, ma non la direzione del moto espressa, come detto sopra, da \(d \overrightarrow{r}\) e quindi da \(\overrightarrow{v}\).
  \item La velocità v è dunque sempre tangente alla traiettoria; l'accelerazione in generale non è parallela alla velocità. La componente di \(\overrightarrow{\alpha}\) ortogonale a \(\overrightarrow{v}\) descrive le variazioni della direzione del moto e non ha alcun legame con le variazioni del modulo di \(\overrightarrow{v}\). Nel moto curvilineo c'è sempre una componente dell'accelerazione ortogonale alla traiettoria, nel moto rettilineo questa componente è sempre nulla.
  \item Da \(r(t)\) si passa a \(\boldsymbol{v}(t)\) e ad \(\boldsymbol{a}(t)\) derivando rispetto al tempo; se invece è data \(\boldsymbol{a}(t)\) il passaggio inverso si fa integrando ed è essenziale la conoscenza delle condizioni iniziali.
  \item Le grandezze cinematiche fondamentali sono dunque posizione, velocità, accelerazione e sono legate tra loro da operazioni di derivazione e di integrazione, conseguenza del fatto che lo studio dell'evoluzione del moto comporta il concetto di variazione che è espresso matematicamente dall'operazione di derivazione.
  \item \(\mathbf{r}, \boldsymbol{v}, \boldsymbol{a}\) sono grandezze vettoriali: nella loro definizione oltre ad un numero bisogna dare una direzione e un verso; pertanto le variazioni di posizione e di velocità sono espresse completamente solo se si tiene conto sia della variazione del modulo che della variazione della direzione.
  \item È necessario specificare sempre in quale sistema di riferimento si descrive il moto: le coordinate del punto, le componenti di \(\boldsymbol{v}\) e \(\boldsymbol{a}\), l'espressione analitica della traiettoria dipendono dal sistema di riferimento. Però le relazioni più generali tra le grandezze cinematiche sono relazioni vettoriali e in quanto tali sono invarianti rispetto alla scelta del sistema di riferimento, cioè valgono in qualsiasi sistema di riferimento.
  Quindi, se \(\mathbf{b}=\mathbf{c}\), questo comporta sempre \(b_{x}=c_{x}, b_{y}=c_{y}, b_{z}=c_{z}\) ma i valori \(\left(b_{x}, b_{y}, b_{z}\right)\) e \(\left(c_{x}, c_{y}, c_{z}\right)\) cambiano col sistema di riferimento.
\end{itemize}
\end{document}
