\documentclass[class=book, crop=false, oneside, 12pt]{standalone}
\usepackage{standalone}
\usepackage{amsmath}
\usepackage{../../style}
% \graphicspath{{./assets/images/}}

% arara: pdflatex: { synctex: yes, shell: yes }
% arara: latexmk: { clean: partial }
\begin{document}

\chapter{Conduttori, dielettrici, energia elettrostatica}

\section{Conduttori in equilibrio}

\subsection{Condizione di equilibrio per un conduttore}

I materiali conduttori sono caratterizzati dal fatto che nel loro interno sono verificate particolari condizioni per cui è possibile il moto di alcune delle cariche che li costituiscono.  
La nostra trattazione si concentra sui conduttori solidi, il cui esempio più tipico sono i metalli: in essi per ogni atomo si hanno uno o più elettroni che sono in pratica separati dal resto dell'atomo e liberi di muoversi nel conduttore. 
Con l'applicazione di un opportuno campo \(\overrightarrow{E}\) si può provocare un moto ordinato di elettroni ovvero dar luogo a una corrente elettrica.

Nei fenomeni elettrostatici però le cariche sono fisse e questa condizione richiede che all'interno di un conduttore il campo debba essere nullo, altrimenti ci sarebbe un moto di cariche, contrariamente all'ipotesi. 
Pertanto lo stato di conduttore in equilibrio elettrostatico è definito dalla condizione: 
\begin{equation*}
    \overrightarrow{E} = 0 \text{ all'interno }
\end{equation*}
Si deve intendere che questa è una condizione media macroscopica. 
Nelle immediate vicinanze dei nuclei ci sono campi molto intensi che tengono legati gli elettroni non liberi; inoltre gli elettroni liberi non sono in quiete ma hanno un moto completamente disordinato di agitazione termica. 
Però in nessun istante c'è un moto ordinato in una certa direzione degli elettroni liberi rispetto agli ioni metallici fissi; si usa per questo parlare di gas di elettroni liberi all'interno di un conduttore.

La condizione \(\overrightarrow{E} = 0\) ha le seguenti conseguenze che caratterizzano un conduttore in equilibrio elettrostatico: 
\begin{itemize}
    \item l'eccesso di carica elettrica in un conduttore può stare solo sulla superficie del conduttore;
    \item il potenziale elettrostatico è costante su tutto il conduttore;
    \item il campo elettrostatico in un punto delle vicinanze della superficie del conduttore è perpendicolare alla superficie e ha intensità \(\sigma/ \epsilon_0\), con \(\sigma\) densità di carica superficiale in quel punto.
\end{itemize}
Per la prima proprietà, se il campo elettrostatico è nullo, è nullo il flusso attraverso una qualunque superficie chiusa \(\Sigma^{\prime}\) tracciata all'interno del conduttore e quindi secondo la legge di Gauss all'interno del conduttore non ci sono cariche ( \(q_{int} = 0\)).
Pertanto l'eccesso di carica si distribuisce sulla superficie del conduttore con densità superficiale \(\sigma = dq/ d \Sigma\); se si cedono al conduttore elettroni questi si portano sulla superficie, se si sottraggono, ne risulta carente lo strato superficiale. 

Il potenziale elettrostatico risulta costante in ogni punto del conduttore perché presi due punti qualsiasi
\begin{equation*}
    V(P_2) - V(P_1) = - \int_{P_1}^{P_2} \overrightarrow{E} \cdot d \overrightarrow{s} = 0 \implies V(P_2) = V(P_1) = V_0
\end{equation*}
Il risultato è vero anche se uno dei due punti sta sulla superficie del conduttore, che risulta quindi essere una \emph{superficie equipotenziale}. 

\subsection{Teorema di Coulomb}

Dato che la superficie del conduttore è equipotenziale, il campo elettrostatico \(\overrightarrow{E}\) in un punto esterno molto vicino al conduttore è ortogonale alla superficie del conduttore, indipendentemente dalla forma di questo.

Il valore di \(\overrightarrow{E}\) si ricava applicando la legge di Gauss ad un cilindro retto di basi \(d \Sigma\) e superficie laterale di area trascurabile rispetto a \(d \Sigma\), con una base contenuta all'interno del conduttore, in cui \(E = 0\), e l'altra in prossimità immediata del conduttore all'esterno, dove il campo elettrostatico \(\overrightarrow{E}\) è normale alla superficie. 
Detta \(dq\) la carica contenuta all'interno, sulla superficie del conduttore, si ha:
\begin{equation*}
    \oint \overrightarrow{E} \cdot \overrightarrow{u}_n d \Sigma = E d \Sigma = \frac{1}{\epsilon_0} d q = \frac{1}{\epsilon_0} \sigma d \Sigma
\end{equation*}
e quindi
\begin{equation} \label{teorema_coulomb}
    \overrightarrow{E} = \frac{\sigma}{\epsilon_0}\overrightarrow{u}_n
\end{equation}
detto anche \emph{Teorema di Coulomb}. 
Il verso è uscente se la densità è positiva, entrante se è negativa. 
Si vede che il modulo del campo elettrostatico è maggiore dove \(\sigma\) è maggiore (questo avviene dove il raggio di curvatura è minore).

Un conduttore carico lontano da altri conduttori ha dunque una distribuzione superficiale di carica tale che il campo elettrostatico all'interno sia nullo, qualunque sia la forma del conduttore. 
In particolare se il conduttore è sferico la carica è distribuita uniformemente.
Notiamo inoltre che la carica deve avere lo stesso segno, positivo o negativo, ovunque sulla superficie: un accumulo di elettroni soltanto in una certa zona sarebbe dovuto esclusivamente a un campo elettrico esterno.

\subsection{Induzione elettrostatica}

Avvicinando un conduttore, carico o scarico, ad un altro corpo carico, ovvero introducendolo in un campo elettrico esterno \(\overrightarrow{E}\), il campo elettrostatico all'interno non sarebbe più nullo, ma sarebbe dato da \(\overrightarrow{E}\); senonché questo fatto provoca un movimento di elettroni che si spostano per l'azione del campo elettrico esterno e si accumulano in una zona della superficie, lasciando sul resto della superficie un eccesso di carica positiva: 
tra queste zone si crea un campo dettrostatico indotto \(\overrightarrow{E}_i\) che contrasta il movimento degli elettroni e si raggiunge l'equilibrio quando \(E + E_i = 0\) in tutto l'interno del conduttore.
Abbiamo così una distribuzione di carica elettrica indotta dei due segni sulla superficie del conduttore che si sovrappone all'eventuale carica elettrica preesistente; in totale però la carica elettrica del conduttore rimane la stessa poiché la carica elettrica indotta è la somma algebrica dei due contributi eguali ed opposti.
È il caso dell'induzione elettrostatica.
%TODO decidere se aggiungere esempi di questo caso

Se poniamo a contatto due o più conduttori, ad esempio collegandoli con un filo conduttore, si costituisce un unico corpo conduttore e in equilibrio vale ovunque la condizione \(E = 0\), \(V =\text{ costante}\): i conduttori a contatto hanno lo stesso potenziale. 


\end{document}
