\documentclass[class=book, crop=false, oneside, 12pt]{standalone}
\usepackage{standalone}
\usepackage{../../style}

% arara: pdflatex: { synctex: yes, shell: yes }
% arara: latexmk: { clean: partial }
\begin{document}
\chapter*{Prefazione}
\addcontentsline{toc}{chapter}{Prefazione}

\section*{Il progetto}
Quella che segue è una dispensa di appunti scritta da studenti; lo scopo è quello di raccogliere i contenuti del corso di Fisica tenuto dal professor Iuppa e condensarli in un elaborato di più facile fruizione per lo studente che si appresta a questa avventura a dir poco epica.

La dispensa fa riferimento al programma svolto durante l'anno accademico 2020/2021 e pertanto non comprende gran parte degli elementi di elettromagnetismo altrimenti affrontati durante il naturale svolgimento del corso.

I contenuti provengono principalmente dalle lezioni del professore, mentre invece ordine ed esposizione sono derivati in gran parte dal Mazzoldi\footnote{\emph{"Fisica vol.1 Meccanica, termodinamica"} e \emph{"Fisica vol.2 Elettromagnetismo, onde"} di \emph{P.Mazzoldi, M.Nigro e C.Voci}}. 

\section*{Istruzioni per l'uso}
Preme sottolineare che non ci assumiamo alcuna responsabilità circa l'esito dell'esame di chi sceglie di utilizzare questo testo come principale risorsa di studio. Nonostante il sangue versato non possiamo ignorare di essere dei poveri studenti, proni a commettere errori o malinterpretare dei passaggi.

Pertanto ci teniamo a consigliare ai lettori di fare sempre riferimento a quanto detto in classe dal professore e al materiale da lui indicato.

\section*{Segnalazione errori}
Come accennato in precedenza, la dispensa è stata scritta durante il secondo semestre dell'anno accademico 2020/2021 e quindi, come molti di voi sapranno, in tempo di pandemia: non vuole essere una scusa, ma solo un modo carino per dirvi che molto probabilmente troverete \emph{innumerevoli} errori dovuti principalmente ai principi di burnout\footnote{Ora stiamo tutti bene. Grazie per l'interessamento.} che abbiamo affrontato durante la stesura del testo. Sono state svolte alcune revisioni, ma siamo comunque certi di aver trascurato qualcosa.

A questo proposito, se durante la lettura doveste incorrere in errori di qualsiasi tipo, tra gli altri errori di battitura, errori concettuali o di impaginazione, vi chiediamo di fare una segnalazione; ve ne saremo riconoscenti e provvederemo a correggere quanto prima.

Per fare ciò potete scegliere uno delle seguenti modalità:
\begin{itemize}
    \item visitando la repository del progetto (la trovate qui: \url{https://github.com/mgiacopu/appunti-fisica}) e aprendo una Github issue in cui segnalate l'errore o, se avete le mani in pasta con \LaTeX~, anche "forkando" la stessa repository e proponendo un vostro fix con una pull request sulla branch \emph{errata-corrige} con codice versione più alto;
    \item se ci conoscete personalmente, contattateci tramite canali informali come Telegram o WhatsApp;
    \item se siete a Trento potete trovare alcuni di noi, con cadenza giornaliera, in biblioteca, da \emph{Panetti Povo\texttrademark} o al bar dalla Emma.
\end{itemize}

\section*{Ringraziamenti}
Di seguito vogliamo ringraziare chi ha contribuito in maniera più o meno diretta alla realizzazione di questo progetto:

\paragraph{La squadra degli Appunti di LFC} per aver fornito la base di questo progetto che ci ha permesso di partire subito all'attacco. Potete trovarli qua: \url{https://github.com/filippodaniotti/Appunti-LFC}\footnote{Il prossimo semestre la loro dispensa vi sarà estremamente utile. Tenetela d'occhio!}

\end{document}