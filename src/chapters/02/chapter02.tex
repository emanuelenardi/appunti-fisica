\documentclass[class=book, crop=false, oneside, 12pt]{standalone}
\usepackage{standalone}
\usepackage{amsmath}
\usepackage{../../style}
% \graphicspath{{./assets/images/}}

% arara: pdflatex: { synctex: yes, shell: yes }
% arara: latexmk: { clean: partial }
\begin{document}

\chapter{Dinamica del punto}

\section{Legge d'inerzia}
L'interazione del punto con l'ambiente circostante, espressa dal concetto di forza, determina la variazione dello stato di moto.

Galileo, da prove sperimentali, formulò il principio d'inerzia: \emph{un corpo non soggetto a forze rimane nel suo stato di moto}. 
Questo venne riformulato da Newton come prima legge della dinamica: \emph{un corpo permane nel suo stato di quiete o di moto rettilineo uniforme a meno che non intervenga una forza esterna a modificare tale stato}. 
Quindi un corpo può essere in stato di quiete \(v = 0\) oppure in moto (Moto Rettilieo Uniforme) con \(v = costante\).

Dal principio di inerzia si deduce che l'interazione, ossia l'azione di una forza, porta a una variazione della velocità in modulo o in direzione o in entrambi.
La forza dunque è la grandezza che esprime e misura l'interazione tra sistemi fisici. 

Alla forza è associata la nozione di intensità e di direzionalità.
L'effetto di una forza cambia con la direzione. Se più forze agiscono sullo stesso corpo ma il corpo non va in uno stato di moto (ovvero la risultante delle forze è nulla) si dice ce si trova in uno stato di equilibrio.

\section{Legge di Newton }
La formulazione quantitativa del legame tra la forza e lo stato di moto è data dalla legge di Newton:
\begin{equation}
    \overrightarrow{F} = m \overrightarrow{a}
\end{equation}
L'interazione del punto con l'ambiente circostante, espressa tramite la forza \(F\), determina l'accelerazione del punto ovvero la variazione della sua velocità nel tempo; \(m\) rappresenta la massa inerziale del punto (Seconda legge della dinamica).

Il termine massa inerziale è legato al fatto che la massa esprime l'inerzia del punto, cioè la sua resistenza a variare il proprio stato di moto, ossia a modificare la velocità (in modulo, direzione e verso).
Nello studio del corpo, che è sempre rappresentato come punto materiale, possiamo trascurarare le sue dimensioni ma non la sua massa che diventa fondamentale.

La legge può essere espansa come:
\begin{equation}
    \overrightarrow{F} = m \frac{\overrightarrow{v}}{dt} = m \frac{d^2 \overrightarrow{r}}{dt^2}
\end{equation}
da questa possiamo ricavare tutte le proprietà al moto di un punto materiale. Viceversa è possibile dall'accelerazione determinare la forza agente e la massa.

Le leggi di Newton sono valide solo se il moto è studiato in una particolare classe di sistemi di riferimento, i cosiddetti sistemi di riferimento inerziali; altrimenti compaiono nelle formule termini correttivi.
In breve un sistema di riferimento inerziale è un sistema che si muove in moto rettilineo uniforme rispetto a un altro sistema, ossia con velocità costante rispetto ad esso.

\end{document}