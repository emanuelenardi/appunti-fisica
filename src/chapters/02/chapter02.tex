\documentclass[class=book, crop=false, oneside, 12pt]{standalone}
\usepackage{standalone}
\usepackage{amsmath}
\usepackage{../../style}
% \graphicspath{{./assets/images/}}

% arara: pdflatex: { synctex: yes, shell: yes }
% arara: latexmk: { clean: partial }
\begin{document}

\chapter{Dinamica del punto}

\section{Legge d'inerzia}
L'interazione del punto con l'ambiente circostante, espressa dal concetto di forza, determina la variazione dello stato di moto.

Galileo, da prove sperimentali, formulò il principio d'inerzia: \emph{un corpo non soggetto a forze rimane nel suo stato di moto}. 
Questo venne riformulato da Newton come prima legge della dinamica: \emph{un corpo permane nel suo stato di quiete o di moto rettilineo uniforme a meno che non intervenga una forza esterna a modificare tale stato}. 
Quindi un corpo può essere in stato di quiete \(v = 0\) oppure in moto (Moto Rettilieo Uniforme) con \(v = costante\).

Dal principio di inerzia si deduce che l'interazione, ossia l'azione di una forza, porta a una variazione della velocità in modulo o in direzione o in entrambi.
La forza dunque è la grandezza che esprime e misura l'interazione tra sistemi fisici. 

Alla forza è associata la nozione di intensità e di direzionalità.
L'effetto di una forza cambia con la direzione. Se più forze agiscono sullo stesso corpo ma il corpo non va in uno stato di moto (ovvero la risultante delle forze è nulla) si dice ce si trova in uno stato di equilibrio.

\end{document}