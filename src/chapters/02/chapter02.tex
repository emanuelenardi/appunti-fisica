\documentclass[class=book, crop=false, oneside, 12pt]{standalone}
\usepackage{standalone}
\usepackage{amsmath}
\usepackage{../../style}
% \graphicspath{{./assets/images/}}

% arara: pdflatex: { synctex: yes, shell: yes }
% arara: latexmk: { clean: partial }
\begin{document}

\chapter{Dinamica del punto}

\section{Legge d'inerzia}
L'interazione del punto con l'ambiente circostante, espressa dal concetto di forza, determina la variazione dello stato di moto.

Galileo, da prove sperimentali, formulò il principio d'inerzia: \emph{un corpo non soggetto a forze rimane nel suo stato di moto}. 
Questo venne riformulato da Newton come prima legge della dinamica: \emph{un corpo permane nel suo stato di quiete o di moto rettilineo uniforme a meno che non intervenga una forza esterna a modificare tale stato}. 
Quindi un corpo può essere in stato di quiete \(v = 0\) oppure in moto (Moto Rettilieo Uniforme) con \(v = costante\).

Dal principio di inerzia si deduce che l'interazione, ossia l'azione di una forza, porta a una variazione della velocità in modulo o in direzione o in entrambi.
La forza dunque è la grandezza che esprime e misura l'interazione tra sistemi fisici. 

Alla forza è associata la nozione di intensità e di direzionalità.
L'effetto di una forza cambia con la direzione. Se più forze agiscono sullo stesso corpo ma il corpo non va in uno stato di moto (ovvero la risultante delle forze è nulla) si dice ce si trova in uno stato di equilibrio.

\section{Legge di Newton }
La formulazione quantitativa del legame tra la forza e lo stato di moto è data dalla legge di Newton:
\begin{equation}
    \overrightarrow{\mathbf{F}} = m \overrightarrow{\mathbf{a}} \ \left[N\right]
\end{equation}
L'interazione del punto con l'ambiente circostante, espressa tramite la forza \(F\), determina l'accelerazione del punto ovvero la variazione della sua velocità nel tempo; \(m\) rappresenta la massa inerziale del punto (Seconda legge della dinamica).

Il termine massa inerziale è legato al fatto che la massa esprime l'inerzia del punto, cioè la sua resistenza a variare il proprio stato di moto, ossia a modificare la velocità (in modulo, direzione e verso).
Nello studio del corpo, che è sempre rappresentato come punto materiale, possiamo trascurarare le sue dimensioni ma non la sua massa che diventa fondamentale.

La legge può essere espansa come:
\begin{equation}
    \overrightarrow{\mathbf{F}} = m \frac{\overrightarrow{\mathbf{v}}}{dt} = m \frac{d^2 \overrightarrow{\mathbf{r}}}{dt^2} \ \left[N\right]
\end{equation}
da questa possiamo ricavare tutte le proprietà al moto di un punto materiale. Viceversa è possibile dall'accelerazione determinare la forza agente e la massa.

Le leggi di Newton sono valide solo se il moto è studiato in una particolare classe di sistemi di riferimento, i cosiddetti sistemi di riferimento inerziali; altrimenti compaiono nelle formule termini correttivi.
In breve un sistema di riferimento inerziale è un sistema che si muove in moto rettilineo uniforme rispetto a un altro sistema, ossia con velocità costante rispetto ad esso.

La legge di Newton può essere anche riscritta in maniera più estesa come:
\begin{equation}
    \overrightarrow{\mathbf{F}}=\left(\frac{\mathrm{d} m}{\mathrm{~d} t}\right) \mathbf{v}+m\left(\frac{\mathrm{d} \mathbf{v}}{\mathrm{d} t}\right)=\left(\frac{\mathrm{d} m}{\mathrm{~d} t}\right) \mathbf{v}+m \mathbf{a} \ \left[N\right]
\end{equation}
per valutare anche la variazione di massa che si ha lungo il moto.

\section{Quantità di moto}
Si definisce \emph{quantità di moto} di un punto materiale il vettore:
\begin{equation}
    \overrightarrow{\mathbf{p}} = m \overrightarrow{\mathbf{v}} \ \left[kg \cdot \frac{m}{s}\right]
\end{equation}
e ci dice quanta massa è in movimento e di quanto si muove.
Se la massa è costante, l'equazione della seconda legge di Newton si può riscrivere come:
\begin{equation}
    \overrightarrow{\mathbf{F}} = \frac{d \overrightarrow{\mathbf{p}}}{dt} \ \left[N\right]
\end{equation}

Questa relazione è la forma più generale della legge di Newton, utilizzabile anche se la massa non è costante. 
La massa infatti può cambiare in due casi:
\begin{itemize}
    \item modifica della massa di un sistema macroscopico( approssimabile a punto materiale) come avviene per esempio in un veicolo a motore che brucia carburante o in una scala mobile; 
    \item dipendenza della massa dalla velocità: secondo la teoria della relatività ristretta la massa varia al variare della sua velocità (la variazione di massa è rilevante solo a velocità prossime a \(c\))
\end{itemize}

È utile, quando si parla di quantità di moto, fare riferimento anche alla sua relazione rispetto alla forza.
Dall' equazione precedente possiamo ottenere \(\overrightarrow{\mathbf{F}} d t = d \overrightarrow{\mathbf{p}}\): vediamo che l'azione di una forza durante un tempo \(d t\) (nel quale la forza agisce) provoca una variazione infinitesima della quantità di moto del punto. 
In termini finiti si ha:
\begin{equation}
    \overrightarrow{\mathbf{I}} = \int_0^t \overrightarrow{\mathbf{F}} dt = \int_{p_0}^p d \overrightarrow{\mathbf{p}} = \overrightarrow{\mathbf{p}}-\overrightarrow{\mathbf{p_0}} = \Delta \overrightarrow{\mathbf{p}} \ \left[N \cdot s\right]
\end{equation} che definisce il "Teorema dell'impulso".

Il termine vettoriale \(I\), integrale della forza nel tempo, è chiamato impulso della forza. 
La formula precedente esprime che l'impulso di una forza applicata ad un punto materiale provoca la variazione della sua quantità di moto: con \(m\) costante si ha ovviamente:
\begin{equation}
    \overrightarrow{\mathbf{I}} = m (\overrightarrow{\mathbf{v}} - \overrightarrow{\mathbf{v_0}}) = m \Delta v \ \left[N \cdot s\right]
\end{equation}

La variazione della quantità di moto è tanto maggiore quanto più elevato è il valore dell'impulso ovvero, per una determinata forza costante, quanto maggiore è il tempo in cui agisce la forza.
Il teorema dell'impulso è utilizzabile per calcolare effettivamente \(\Delta \overrightarrow{\mathbf{p}}\) solo se si conosce la funzione \(F (t)\), in questo caso basta calcolare l'integrale.
Se \(F = costante\), si ottiene subito
\begin{equation*}
    \overrightarrow{\mathbf{F}} t = m (\overrightarrow{\mathbf{v}} - \overrightarrow{\mathbf{v_0}}) \left[N\right]
\end{equation*}
che è come misurare l'area di un rettangolo.
Invece se misuriamo \(\Delta \overrightarrow{\mathbf{p}}\) con il teorema della media intergrale possiamo calcolare \(\overrightarrow{\mathbf{F_m}}\) della forza agente nell'intervallo di tempo \(t\).

Quando \(\overrightarrow{\mathbf{F}}\) è nulla, \(\Delta \overrightarrow{\mathbf{p}} = 0\) e pertanto \(\overrightarrow{\mathbf{p}} = costante\) : in assenza di forza applicata la quantità di moto di un punto materiale rimane costante o, la quantità di moto si conserva (principio di inerzia). 

\section{Risultante delle forze}
Su un punto materia le possono agire contemporaneamente più forze: si constata che il moto del punto ha luogo come se agisse una sola forza, 
la risultante vettoria le delle forze applicate al punto
\begin{equation*}
    \overrightarrow{\mathbf{R}} = \overrightarrow{\mathbf{F}}_1+\overrightarrow{\mathbf{F}}_2+ ... + \overrightarrow{\mathbf{F}}_n = \Sigma_i \overrightarrow{\mathbf{F}}_i \ \left[N\right]
\end{equation*}
è questa una conferma della natura vettoriale dell'equazione di Newton.

Nello studio del moto otteniamo informazioni solo sulla risultante delle forze agenti sul punto stesso \(\overrightarrow{\mathbf{R}}\) e non sulle singole forze che concorrono a formare la risultante.
Se \(R = 0\) ed il punto ha velocità nulla, esso rimane in quiete: sono realizzate le condizioni di equilibrio statico del punto. 
Devono quindi essere nulle le componenti stesse della risultante, con riferimento ad un sistema di assi cartesiani.

\subsection{Reazioni vincolari}

Se un corpo, soggetto all'azione di una forza o della risultante non nulla di un'insieme di forze, rimane fermo, dobbiamo dedurre che l'azione della forza provoca una reazione dell'ambiente circostante (reazione vincolare) che si esprime tramite una forza, eguale e contraria alla forza o alla risultante delle forze agenti,
applicata al corpo stesso in modo tale che esso rimanga in quiete.

Si introduce quindi la terza legge di Newton, detta anche legge dell'azione e reazione: 
\emph{se un corpo A esercita una forza sul corpo B, allora il corpo B esercita una forza uguale e contraria}.

Nel caso di un corpo appoggiato su di un tavolo, il corpo è soggetto all'azione di attrazione della terra, perpendicolarmente al piano.
Il tavolo deve produrre, viste le condizioni di quiete del corpo, una forza uguale e contraria alla forza di attrazione terrestre che chiamiamo reazione vincolare \(\overrightarrow{\mathbf{N}}\).

In generale la reazione vincolare non è determinabile a priori, utilizzando una data formula, ma deve essere calcolata caso per caso dall'esame delle condizioni fisiche. 

\section{Classificazione delle forze}

Tra tutti i fenomeni fisici che osserviamo possiamo classificare quattro interazioni fondamentali:
\begin{itemize}
    \item Interazione gravitazionale: descrive l'attrazione reciproca che avviene tra i corpi dotati di massa
    \item Interazione elettromagnetica: descrive le relazioni tra corpi che possiedono carica, il conseguente campo elettromagnetico che va a formatsi e come quest'ultimo si propaga (onde elettromagnetiche). La conciliazione di queste forze ha permesso di descrivere il fenomeno della luce.
    \item Interazione forte o nucleare: tiene uniti gli elementi costituenti dei protoni e neutroni all'interno del nucleo di un atomo. 
    \item Interazione debole: responsabile delle forze che intervengono nei decadimenti radioattivi.
\end{itemize}

\section{Azione dinamica delle forze}

Nel capitolo 1 abbiamo descritto vari moti, ora andiamo a vedere quali forze questi moti vanno a creare.

Nel moto rettilineo uniforme ho \(\overrightarrow{\mathbf{v}} = costante, \overrightarrow{\mathbf{a}} = 0\), quindi la \(\overrightarrow{\mathbf{F}} = 0\).
Posso ottenere questa condizione anche se agiscono più forze, l'importante è che la risultante sia nulla.

Nel moto uniformemente accelerato ho \(\overrightarrow{\mathbf{a}} = costante\), quindi la forza agente è costante, 
questo vale anche al contrario: se mi trovo una forza risultante costante nella direzione parallela al moto ho accelerazione costante.

Se \(\overrightarrow{\mathbf{F}}\) è variabile ho un moto vario, posso quindi spezzare l'accelerazione nelle sue componenti:
\begin{itemize}
    \item \(\overrightarrow{\mathbf{a_T}}\) detta tangenziale è parallela al moto e descrive una variazione in modulo della velocità
    \item \(\overrightarrow{\mathbf{a_N}}\) detta ortogonale o centripeta, descrive una variazione di direzione della velocità
\end{itemize}

Nel moto circolare uniforme ho \(\overrightarrow{\mathbf{a_T}} = 0\) e \(\overrightarrow{\mathbf{a_N}} = costante\).

\section{Forza peso}

La forza peso è l'attrazione che la terra (nel nostro caso) esercita verso tutti i corpi dotati di massa a lei vicini, questa si manifesta trascinando a Terra i corpi. 
Più in generale è una manifestazione dell'iterazione gravitazionale tra due corpi dotati di massa, in questo caso tra la Terra e un corpo lasciato cadere.
La forza che si crea ha modulo verso il suolo che vale \(g = 9.8 ms^{-2}\).

Se agisce solo la forza peso ottengo che \(\overrightarrow{\mathbf{g}} = \overrightarrow{\mathbf{a}}\), 
nel caso in cui è presente l'attrito dell'aria ho un accelerazione negativa rispetto a quella di gravita quindi ho che \(\overrightarrow{\mathbf{g}} \neq \overrightarrow{\mathbf{a}}\)

È importante notare che un corpo presenta due tipi di massa:
\begin{itemize}
    \item la massa inerziale, che descrive la capacità di un corpo a opporsi al moto
    \item la massa gravitazionale che descrive la capacità di un corpo di esercitare un attrazione reciproca ai corpi adiacenti
\end{itemize}
Per tanto tempo questa doppia definizione ha generato quesiti, infatti sperimentalmente sembravano essere proporzionali tra loro ma non si sapeva bene il perchè.
Fu che Einstein, nel principio di equivalenza, enunciò in definitiva che questi due tipi di massa coincidono.

\section{Forza di attrito radente}

Fino ad ora abbiamo trascurato la presenza dell'attrito dell'aria e delle superifici, ma sappiamo bene che in natura questo non avviene.
Tra i vari tipi di attrito iniziamo con l'attrito radente che è dovuto allo strisciamento di superifici piane che scorrono l'una sopra l'altra.

Quando il corpo non è in movimento e gli si applica una forza, questo rimane fermo fino a che la forza non supera il valore \(\mu_s N\), 
dove \(\mu_s\) è il coefficente di attrito statico ed \(N\) è il modulo della componente normale al piano di appoggio della reazione vincolare.
La forza che si oppone gradualmente al moto è chiamata forza di \emph{attrito radente statico}.
La condizione è quindi:
\begin{equation*}
    F > \mu_s N \left[N\right]
\end{equation*}

Quando il corpo entra in movimento la forza di attrito che si oppone la moto è la forza di \emph{attrito radente dinamico} \(F_a = \mu_d N \ \left[N\right]\) dove \(\mu_d\) rappresenta il coefficente di attrito dinamico.
Ho sempre che \(\mu_d < \mu_s\) quindi l'attrito dinamico è sempre minore dell'attrito statico.
Questo coefficente di attrito radente dipende dalle caratteristiche chimiche e dallo stato della superifice dei due materiali che vanno a sfregarsi tra loro (è adimensionale).

La forza di attrito radente, che è una reazione vincolare, non dipende solo dalle due superifici a contatto ma anche da quanto è premuto il corpo contro il piano (qui entra in gioco la componente \(N\)).
Poi è importante notare che le forze di attrito sono sempre presenti per quanto si provi a ridurle.

\section{Forza elastica}

Si definisce \emph{forza elastica} (unidimensionale) una forza di direzione costante, con verso sempre rivolto verso il punto \(O\) chiamato centro.

La legge di Hooke dice che:
\begin{equation}
    \overrightarrow{\mathbf{F_{el}}} = -k x \hat{x} \left[N\right]
\end{equation}
dove \(k\) è una costante positiva, detta \emph{costante elastica} e \(\hat{x}\) il versore della differenza di lunghezza rispetto \(O\).

L'accelerazione vale:
\begin{equation}
    a = \frac{F}{m} = -\frac{k}{m}x = -\omega^2 x \left[m/s^{-2}\right]
\end{equation}
identifico quindi un moto armonico semplice con pulsazione \(\omega = \sqrt{\frac{k}{m}} \left[rad/s\right]\) e periodo \(T = \frac{2 \pi}{\omega} = 2 \pi \sqrt{\frac{m}{k}} \left[s\right]\).

La forza elastica viene applicata tramite una molla, che ha una lunghezza a riposo \(l_0\) che varia se sottoposta a compressione o estensione. 
In questi casi la molla produce una forza (elastica) che cerca di riportarla a riposo, il verso della forza è sempre di verso opposto all'elongazione. 
Il modulo di questa forza \emph{di richiamo} è proporzionale alla deformazione fino a che non si supera il limite di elasticità della molla.
Se voglio mantenere la molla nella posizione desiderata devo applicare una forza uguale e opposta a quella della molla.

Se attacco una massa alla molla e la lascio andare avrò che la massa si muove con moto armonico per effetto della forza elastica su di esso.
Dalla definizione di forza e dalla legge di Hooke ottengo:
\begin{equation*}
    F = -kx = m a = \frac{d^2 x}{dt^2}
\end{equation*}
Quindi vale:
\begin{equation}
    m \frac{d^2 x}{dt^2} + k x = 0  
\end{equation}
che viene detta equazione differenziale dell'oscillatore armonico semplice. 

Risolvendo l'equazione differenziale ottengo la legge oraria che è:
\begin{equation}
    x(t) = A \cos (\omega t + \phi)
\end{equation}
Per verificare la soluzione posso sostituire \(\frac{d^2x}{dt^2}\) a \(x\) e ottengo che l'equazione è valida se \(\omega^2 = \frac{k}{m} \left[rad/s\right]\).
Quindi la pulsazione di questo moto armonico è:
\begin{equation}
    \omega = \sqrt{\frac{k}{m}}
\end{equation}

Derivando ottengo le funzioni di velocità ed accelerazione:
\begin{equation*}
    v = -A \omega sin (\omega t + \phi)
\end{equation*}
\begin{equation*}
    a = - A \omega^2 cos (\omega t + \phi)
\end{equation*}
equazioni medesime al quelle del moto armonico affrontato precedentemente.

Dall'equazione differenziale posso ricavarmi anche il periodo che risulta essere:
\begin{equation*}
    T = \frac{2 \pi}{\omega} \left[s\right]
\end{equation*}

\section{Piano inclinato}

Un piano inclinato ha due caratteristiche fondamentali, un angolo \(\alpha\) che mi indica l'ampiezza del piano e \(\mu_s, \mu_d\) coefficenti di attrito (se questo non viene trascurato).
Quando analizziamo un piano inclinato il nostro sistema di riferimento è cartesiano con l'asse \(x\) parallelo al piano inclinato e l'asse \(y\) ortogonale a quest'ultimo.
Se metto un punto materiale di massa \(m\) su un piano inclinato grazie a questo sistema di riferimento posso scomporre la sua forza peso \(F_p\) nelle componenti \(F_{px}\) e \(F_{py}\).

Nel caso in cui venga trascurato l'attrito ho che nell'asse perpendicolare \(\overrightarrow{\mathbf{F_{py}}} = \overrightarrow{\mathbf{N}}\), 
mentre nell'asse parallelo \(\overrightarrow{\mathbf{F_{px}}} = m \overrightarrow{\mathbf{a}}\). 
Infatti il corpo si muove lungo il piano a meno di forze esterne che ne bilancino il moto.

Nel caso in cui ci sia attrito nell'asse \(y\) la situazione rimane analoga, nell'asse \(x\) invece l'attrito radente potrebbe bilanciare \(\overrightarrow{\mathbf{F_{px}}}\).
Il corpo non inzia il moto fino a che:
\begin{equation*}
    F_{px} = m g \sin \alpha \leq \mu_s N = \mu_s m g \cos \alpha = \mu_s N
\end{equation*}
quindi la condizione per l'equilibrio statico è:
\begin{equation}
    \tan \alpha \leq \mu_s.
\end{equation}

Quando il corpo è in movimento, bisogna fare rifermento alla formula che utilizza l'attrito radente dinamico \(F_{px} = m g \sin \alpha < \mu_d m g \cos \alpha = \mu_d N\).

Se il corpo è in movimento con velocità iniziale \(v_0\):
\begin{itemize}
    \item si ferma se \(\tan \alpha < \mu_d\).
    \item prosegue con velocità \(v_0\) se \(\tan \alpha  = \mu_d\).
    \item prosegue in moto uniformemente accelerato se \(\tan \alpha > \mu_d\).
\end{itemize}
\end{document}