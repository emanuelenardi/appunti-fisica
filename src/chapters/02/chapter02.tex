\documentclass[class=book, crop=false, oneside, 12pt]{standalone}
\usepackage{standalone}
\usepackage{amsmath}
\usepackage{../../style}
% \graphicspath{{./assets/images/}}

% arara: pdflatex: { synctex: yes, shell: yes }
% arara: latexmk: { clean: partial }
\begin{document}

\chapter{Dinamica del punto}

\section{Legge d'inerzia}
L'interazione del punto con l'ambiente circostante, espressa dal concetto di forza, determina la variazione dello stato di moto.

Galileo, da prove sperimentali, formulò il principio d'inerzia: \emph{un corpo non soggetto a forze rimane nel suo stato di moto}. 
Questo venne riformulato da Newton come prima legge della dinamica: \emph{un corpo permane nel suo stato di quiete o di moto rettilineo uniforme a meno che non intervenga una forza esterna a modificare tale stato}. 
Quindi un corpo può essere in stato di quiete \(v = 0\) oppure in moto (Moto Rettilieo Uniforme) con \(v = costante\).

Dal principio di inerzia si deduce che l'interazione, ossia l'azione di una forza, porta a una variazione della velocità in modulo o in direzione o in entrambi.
La forza dunque è la grandezza che esprime e misura l'interazione tra sistemi fisici. 

Alla forza è associata la nozione di intensità e di direzionalità.
L'effetto di una forza cambia con la direzione. Se più forze agiscono sullo stesso corpo ma il corpo non va in uno stato di moto (ovvero la risultante delle forze è nulla) si dice ce si trova in uno stato di equilibrio.

\section{Legge di Newton }
La formulazione quantitativa del legame tra la forza e lo stato di moto è data dalla legge di Newton:
\begin{equation}
    \overrightarrow{\mathbf{F}} = m \overrightarrow{\mathbf{a}} \ \udm{N}
\end{equation}
L'interazione del punto con l'ambiente circostante, espressa tramite la forza \(F\), determina l'accelerazione del punto ovvero la variazione della sua velocità nel tempo; \(m\) rappresenta la massa inerziale del punto (Seconda legge della dinamica).

Il termine massa inerziale è legato al fatto che la massa esprime l'inerzia del punto, cioè la sua resistenza a variare il proprio stato di moto, ossia a modificare la velocità (in modulo, direzione e verso).
Nello studio del corpo, che è sempre rappresentato come punto materiale, possiamo trascurarare le sue dimensioni ma non la sua massa che diventa fondamentale.

La legge può essere espansa come:
\begin{equation}
    \overrightarrow{\mathbf{F}} = m \frac{\overrightarrow{\mathbf{v}}}{dt} = m \frac{d^2 \overrightarrow{\mathbf{r}}}{dt^2} \ \udm{N}
\end{equation}
da questa possiamo ricavare tutte le proprietà al moto di un punto materiale. Viceversa è possibile dall'accelerazione determinare la forza agente e la massa.

Le leggi di Newton sono valide solo se il moto è studiato in una particolare classe di sistemi di riferimento, i cosiddetti sistemi di riferimento inerziali; altrimenti compaiono nelle formule termini correttivi.
In breve un sistema di riferimento inerziale è un sistema che si muove in moto rettilineo uniforme rispetto a un altro sistema, ossia con velocità costante rispetto ad esso.

La legge di Newton può essere anche riscritta in maniera più estesa come:
\begin{equation}
    \overrightarrow{\mathbf{F}}=\left(\frac{\mathrm{d} m}{\mathrm{~d} t}\right) \mathbf{v}+m\left(\frac{\mathrm{d} \mathbf{v}}{\mathrm{d} t}\right)=\left(\frac{\mathrm{d} m}{\mathrm{~d} t}\right) \mathbf{v}+m \mathbf{a} \ \udm{N}
\end{equation}
per valutare anche la variazione di massa che si ha lungo il moto.

\section{Quantità di moto}
Si definisce \emph{quantità di moto} di un punto materiale il vettore:
\begin{equation}
    \overrightarrow{\mathbf{p}} = m \overrightarrow{\mathbf{v}} \ \udm{kg \cdot \frac{m}{s}}
\end{equation}
e ci dice quanta massa è in movimento e di quanto si muove.
Se la massa è costante, l'equazione della seconda legge di Newton si può riscrivere come:
\begin{equation}
    \overrightarrow{\mathbf{F}} = \frac{d \overrightarrow{\mathbf{p}}}{dt} \ \udm{N}
\end{equation}

Questa relazione è la forma più generale della legge di Newton, utilizzabile anche se la massa non è costante. 
La massa infatti può cambiare in due casi:
\begin{itemize}
    \item modifica della massa di un sistema macroscopico( approssimabile a punto materiale) come avviene per esempio in un veicolo a motore che brucia carburante o in una scala mobile; 
    \item dipendenza della massa dalla velocità: secondo la teoria della relatività ristretta la massa varia al variare della sua velocità (la variazione di massa è rilevante solo a velocità prossime a \(c\))
\end{itemize}

È utile, quando si parla di quantità di moto, fare riferimento anche alla sua relazione rispetto alla forza.
Dall' equazione precedente possiamo ottenere \(\overrightarrow{\mathbf{F}} d t = d \overrightarrow{\mathbf{p}}\): vediamo che l'azione di una forza durante un tempo \(d t\) (nel quale la forza agisce) provoca una variazione infinitesima della quantità di moto del punto. 
In termini finiti si ha:
\begin{equation}
    \overrightarrow{\mathbf{I}} = \int_0^t \overrightarrow{\mathbf{F}} dt = \int_{p_0}^p d \overrightarrow{\mathbf{p}} = \overrightarrow{\mathbf{p}}-\overrightarrow{\mathbf{p_0}} = \Delta \overrightarrow{\mathbf{p}} \ \udm{N \cdot s}
\end{equation} che definisce il "Teorema dell'impulso".

Il termine vettoriale \(I\), integrale della forza nel tempo, è chiamato impulso della forza. 
La formula precedente esprime che l'impulso di una forza applicata ad un punto materiale provoca la variazione della sua quantità di moto: con \(m\) costante si ha ovviamente:
\begin{equation}
    \overrightarrow{\mathbf{I}} = m (\overrightarrow{\mathbf{v}} - \overrightarrow{\mathbf{v_0}}) = m \Delta v \ \udm{N \cdot s}
\end{equation}

La variazione della quantità di moto è tanto maggiore quanto più elevato è il valore dell'impulso ovvero, per una determinata forza costante, quanto maggiore è il tempo in cui agisce la forza.
Il teorema dell'impulso è utilizzabile per calcolare effettivamente \(\Delta \overrightarrow{\mathbf{p}}\) solo se si conosce la funzione \(F (t)\), in questo caso basta calcolare l'integrale.
Se \(F = costante\), si ottiene subito
\begin{equation*}
    \overrightarrow{\mathbf{F}} t = m (\overrightarrow{\mathbf{v}} - \overrightarrow{\mathbf{v_0}}) \udm{N}
\end{equation*}
che è come misurare l'area di un rettangolo.
Invece se misuriamo \(\Delta \overrightarrow{\mathbf{p}}\) con il teorema della media intergrale possiamo calcolare \(\overrightarrow{\mathbf{F_m}}\) della forza agente nell'intervallo di tempo \(t\).

Quando \(\overrightarrow{\mathbf{F}}\) è nulla, \(\Delta \overrightarrow{\mathbf{p}} = 0\) e pertanto \(\overrightarrow{\mathbf{p}} = costante\) : in assenza di forza applicata la quantità di moto di un punto materiale rimane costante o, la quantità di moto si conserva (principio di inerzia). 

\section{Risultante delle forze}
Su un punto materia le possono agire contemporaneamente più forze: si constata che il moto del punto ha luogo come se agisse una sola forza, 
la risultante vettoria le delle forze applicate al punto
\begin{equation*}
    \overrightarrow{\mathbf{R}} = \overrightarrow{\mathbf{F}}_1+\overrightarrow{\mathbf{F}}_2+ ... + \overrightarrow{\mathbf{F}}_n = \Sigma_i \overrightarrow{\mathbf{F}}_i \ \udm{N}
\end{equation*}
è questa una conferma della natura vettoriale dell'equazione di Newton.

Nello studio del moto otteniamo informazioni solo sulla risultante delle forze agenti sul punto stesso \(\overrightarrow{\mathbf{R}}\) e non sulle singole forze che concorrono a formare la risultante.
Se \(R = 0\) ed il punto ha velocità nulla, esso rimane in quiete: sono realizzate le condizioni di equilibrio statico del punto. 
Devono quindi essere nulle le componenti stesse della risultante, con riferimento ad un sistema di assi cartesiani.

\subsection{Reazioni vincolari}

Se un corpo, soggetto all'azione di una forza o della risultante non nulla di un'insieme di forze, rimane fermo, dobbiamo dedurre che l'azione della forza provoca una reazione dell'ambiente circostante (reazione vincolare) che si esprime tramite una forza, eguale e contraria alla forza o alla risultante delle forze agenti,
applicata al corpo stesso in modo tale che esso rimanga in quiete.

Si introduce quindi la terza legge di Newton, detta anche legge dell'azione e reazione: 
\emph{se un corpo A esercita una forza sul corpo B, allora il corpo B esercita una forza uguale e contraria}.

Nel caso di un corpo appoggiato su di un tavolo, il corpo è soggetto all'azione di attrazione della terra, perpendicolarmente al piano.
Il tavolo deve produrre, viste le condizioni di quiete del corpo, una forza uguale e contraria alla forza di attrazione terrestre che chiamiamo reazione vincolare \(\overrightarrow{\mathbf{N}}\).

In generale la reazione vincolare non è determinabile a priori, utilizzando una data formula, ma deve essere calcolata caso per caso dall'esame delle condizioni fisiche. 

\section{Classificazione delle forze}

Tra tutti i fenomeni fisici che osserviamo possiamo classificare quattro interazioni fondamentali:
\begin{itemize}
    \item Interazione gravitazionale: descrive l'attrazione reciproca che avviene tra i corpi dotati di massa
    \item Interazione elettromagnetica: descrive le relazioni tra corpi che possiedono carica, il conseguente campo elettromagnetico che va a formatsi e come quest'ultimo si propaga (onde elettromagnetiche). La conciliazione di queste forze ha permesso di descrivere il fenomeno della luce.
    \item Interazione forte o nucleare: tiene uniti gli elementi costituenti dei protoni e neutroni all'interno del nucleo di un atomo. 
    \item Interazione debole: responsabile delle forze che intervengono nei decadimenti radioattivi.
\end{itemize}

\section{Azione dinamica delle forze}

Nel capitolo 1 abbiamo descritto vari moti, ora andiamo a vedere quali forze questi moti vanno a creare.

Nel moto rettilineo uniforme ho \(\overrightarrow{\mathbf{v}} = costante, \overrightarrow{\mathbf{a}} = 0\), quindi la \(\overrightarrow{\mathbf{F}} = 0\).
Posso ottenere questa condizione anche se agiscono più forze, l'importante è che la risultante sia nulla.

Nel moto uniformemente accelerato ho \(\overrightarrow{\mathbf{a}} = costante\), quindi la forza agente è costante, 
questo vale anche al contrario: se mi trovo una forza risultante costante nella direzione parallela al moto ho accelerazione costante.

Se \(\overrightarrow{\mathbf{F}}\) è variabile ho un moto vario, posso quindi spezzare l'accelerazione nelle sue componenti:
\begin{itemize}
    \item \(\overrightarrow{\mathbf{a_T}}\) detta tangenziale è parallela al moto e descrive una variazione in modulo della velocità
    \item \(\overrightarrow{\mathbf{a_N}}\) detta ortogonale o centripeta, descrive una variazione di direzione della velocità
\end{itemize}

Nel moto circolare uniforme ho \(\overrightarrow{\mathbf{a_T}} = 0\) e \(\overrightarrow{\mathbf{a_N}} = costante\).

\section{Forza peso}

La forza peso è l'attrazione che la terra (nel nostro caso) esercita verso tutti i corpi dotati di massa a lei vicini, questa si manifesta trascinando a Terra i corpi. 
Più in generale è una manifestazione dell'iterazione gravitazionale tra due corpi dotati di massa, in questo caso tra la Terra e un corpo lasciato cadere.
La forza che si crea ha modulo verso il suolo che vale \(g = 9.8 ms^{-2}\).

Se agisce solo la forza peso ottengo che \(\overrightarrow{\mathbf{g}} = \overrightarrow{\mathbf{a}}\), 
nel caso in cui è presente l'attrito dell'aria ho un accelerazione negativa rispetto a quella di gravita quindi ho che \(\overrightarrow{\mathbf{g}} \neq \overrightarrow{\mathbf{a}}\)

È importante notare che un corpo presenta due tipi di massa:
\begin{itemize}
    \item la massa inerziale, che descrive la capacità di un corpo a opporsi al moto
    \item la massa gravitazionale che descrive la capacità di un corpo di esercitare un attrazione reciproca ai corpi adiacenti
\end{itemize}
Per tanto tempo questa doppia definizione ha generato quesiti, infatti sperimentalmente sembravano essere proporzionali tra loro ma non si sapeva bene il perchè.
Fu che Einstein, nel principio di equivalenza, enunciò in definitiva che questi due tipi di massa coincidono.

\section{Forza di attrito radente}

Fino ad ora abbiamo trascurato la presenza dell'attrito dell'aria e delle superifici, ma sappiamo bene che in natura questo non avviene.
Tra i vari tipi di attrito iniziamo con l'attrito radente che è dovuto allo strisciamento di superifici piane che scorrono l'una sopra l'altra.

Quando il corpo non è in movimento e gli si applica una forza, questo rimane fermo fino a che la forza non supera il valore \(\mu_s N\), 
dove \(\mu_s\) è il coefficente di attrito statico ed \(N\) è il modulo della componente normale al piano di appoggio della reazione vincolare.
La forza che si oppone gradualmente al moto è chiamata forza di \emph{attrito radente statico}.
La condizione è quindi:
\begin{equation*}
    F > \mu_s N \udm{N}
\end{equation*}

Quando il corpo entra in movimento la forza di attrito che si oppone la moto è la forza di \emph{attrito radente dinamico} \(F_a = \mu_d N \ \udm{N}\) dove \(\mu_d\) rappresenta il coefficente di attrito dinamico.
Ho sempre che \(\mu_d < \mu_s\) quindi l'attrito dinamico è sempre minore dell'attrito statico.
Questo coefficente di attrito radente dipende dalle caratteristiche chimiche e dallo stato della superifice dei due materiali che vanno a sfregarsi tra loro (è adimensionale).

La forza di attrito radente, che è una reazione vincolare, non dipende solo dalle due superifici a contatto ma anche da quanto è premuto il corpo contro il piano (qui entra in gioco la componente \(N\)).
Poi è importante notare che le forze di attrito sono sempre presenti per quanto si provi a ridurle.

\section{Forza elastica}

Si definisce \emph{forza elastica} (unidimensionale) una forza di direzione costante, con verso sempre rivolto verso il punto \(O\) chiamato centro.

La legge di Hooke dice che:
\begin{equation}
    \overrightarrow{\mathbf{F_{el}}} = -k x \hat{x} \udm{N}
\end{equation}
dove \(k\) è una costante positiva, detta \emph{costante elastica} e \(\hat{x}\) il versore della differenza di lunghezza rispetto \(O\).

L'accelerazione vale:
\begin{equation}
    a = \frac{F}{m} = -\frac{k}{m}x = -\omega^2 x \udm{m/s^{-2}}
\end{equation}
identifico quindi un moto armonico semplice con pulsazione \(\omega = \sqrt{\frac{k}{m}} \udm{rad/s}\)  e periodo \(T = \frac{2 \pi}{\omega} = 2 \pi \sqrt{\frac{m}{k}} \udm{s}\).

La forza elastica viene applicata tramite una molla, che ha una lunghezza a riposo \(l_0\) che varia se sottoposta a compressione o estensione. 
In questi casi la molla produce una forza (elastica) che cerca di riportarla a riposo, il verso della forza è sempre di verso opposto all'elongazione. 
Il modulo di questa forza \emph{di richiamo} è proporzionale alla deformazione fino a che non si supera il limite di elasticità della molla.
Se voglio mantenere la molla nella posizione desiderata devo applicare una forza uguale e opposta a quella della molla.

Se attacco una massa alla molla e la lascio andare avrò che la massa si muove con moto armonico per effetto della forza elastica su di esso.
Dalla definizione di forza e dalla legge di Hooke ottengo:
\begin{equation*}
    F = -kx = m a = \frac{d^2 x}{dt^2}
\end{equation*}
Quindi vale:
\begin{equation}
    m \frac{d^2 x}{dt^2} + k x = 0  
\end{equation}
che viene detta equazione differenziale dell'oscillatore armonico semplice. 

Risolvendo l'equazione differenziale ottengo la legge oraria che è:
\begin{equation}
    x(t) = A \cos (\omega t + \phi)
\end{equation}
Per verificare la soluzione posso sostituire \(\frac{d^2x}{dt^2}\) a \(x\) e ottengo che l'equazione è valida se \(\omega^2 = \frac{k}{m} \udm{rad/s}\).
Quindi la pulsazione di questo moto armonico è:
\begin{equation}
    \omega = \sqrt{\frac{k}{m}}
\end{equation}

Derivando ottengo le funzioni di velocità ed accelerazione:
\begin{equation*}
    v = -A \omega sin (\omega t + \phi)
\end{equation*}
\begin{equation*}
    a = - A \omega^2 cos (\omega t + \phi)
\end{equation*}
equazioni medesime al quelle del moto armonico affrontato precedentemente.

Dall'equazione differenziale posso ricavarmi anche il periodo che risulta essere:
\begin{equation*}
    T = \frac{2 \pi}{\omega} \udm{s}
\end{equation*}

\section{Piano inclinato}

Un piano inclinato ha due caratteristiche fondamentali, un angolo \(\alpha\) che mi indica l'ampiezza del piano e \(\mu_s, \mu_d\) coefficenti di attrito (se questo non viene trascurato).
Quando analizziamo un piano inclinato il nostro sistema di riferimento è cartesiano con l'asse \(x\) parallelo al piano inclinato e l'asse \(y\) ortogonale a quest'ultimo.
Se metto un punto materiale di massa \(m\) su un piano inclinato grazie a questo sistema di riferimento posso scomporre la sua forza peso \(F_p\) nelle componenti \(F_{px}\) e \(F_{py}\).

Nel caso in cui venga trascurato l'attrito ho che nell'asse perpendicolare \(\overrightarrow{\mathbf{F_{py}}} = \overrightarrow{\mathbf{N}}\), 
mentre nell'asse parallelo \(\overrightarrow{\mathbf{F_{px}}} = m \overrightarrow{\mathbf{a}}\). 
Infatti il corpo si muove lungo il piano a meno di forze esterne che ne bilancino il moto.

Nel caso in cui ci sia attrito nell'asse \(y\) la situazione rimane analoga, nell'asse \(x\) invece l'attrito radente potrebbe bilanciare \(\overrightarrow{\mathbf{F_{px}}}\).
Il corpo non inzia il moto fino a che:
\begin{equation*}
    F_{px} = m g \sin \alpha \leq \mu_s N = \mu_s m g \cos \alpha = \mu_s N
\end{equation*}
quindi la condizione per l'equilibrio statico è:
\begin{equation}
    \tan \alpha \leq \mu_s.
\end{equation}

Quando il corpo è in movimento, bisogna fare rifermento alla formula che utilizza l'attrito radente dinamico \(F_{px} = m g \sin \alpha < \mu_d m g \cos \alpha = \mu_d N\).

Se il corpo è in movimento con velocità iniziale \(v_0\):
\begin{itemize}
    \item si ferma se \(\tan \alpha < \mu_d\).
    \item prosegue con velocità \(v_0\) se \(\tan \alpha  = \mu_d\).
    \item prosegue in moto uniformemente accelerato se \(\tan \alpha > \mu_d\).
\end{itemize}

\section{Forze centripete}

Supponiamo che la risultante \(\overrightarrow{R}\) delle forze agenti su un punto materiale presenti una componente \(F_N\) ortogonale alla traiettoria, che risulta pertanto curvilinea. \(F_N\) determina l'accelerazione centripeta secondo la relazione \(F = m a_N = m \frac {v^2} {r} \) essendo \(r\) il raggio di curvatura della traiettoria.

In generale \(\overrightarrow{R}\) ha anche una componente tangente alla traiettoria, \(F_T\), responsabile della variazione del modulo della velocità. Se \(F_T = 0\) il moto lungo la traiettoria è uniforme e l'unica accelerazione è \(a_N\)

Una caratteristica comune è l'indipendenza dalla massa del punto delle varie condizioni trovate (velocità in una curva, pendolo conico ecc. ).

Questo perché in tutti gli esempi è presente soltanto la forza peso, che è proporzionale alla massa, e le reazioni vincolari, determinate dall'azione del peso e quindi anch'esse proporzionali alla massa del punto; eguagliando la risultante di tulle le forze \(m \overrightarrow{a}\) la massa viene semplificata e quindi il risultato cinematico è indipendente dalla massa.

Introduciamo anche l'equilibrio dinamico; a differenza dell'equilibrio statico (risultante delle forze applicate al punto eguale a zero, velocità nulla), ci riferiamo a quei particolari casi in cui in presenza di forze il moto avviene con velocità costante in modulo. 

Se si tratta di moto rettilineo ciò è possibile solo se la risultante delle forze è nulla.
Se invece il moto è curvilineo basta che sia nulla \(\overrightarrow{F_T}\) come deve essere se vogliamo che la velocità sia costante in modulo, ovvero che la risultante delle forze agenti sia puramente centripeta.
Qualora \(\overrightarrow{F_N}\) sia anche costante in modulo il moto è circolare uniforme. 

\section{Pendolo semplice}

Il pendolo semplice è costituito da un punto materiale appeso tramite un filo inestensibile e di massa trascurabile. 
La posizione di equilibrio statico è quella verticale, con il punto fermo ed il filo teso; la forza esercitata dal filo (tensione del filo) vale in modulo \(T_F = mg\).

Se spostiamo il punto dalla verticale esso inizia ad oscillare attorno a questa, lungo un arco di circonferenza di raggio \(L\).
Le forze agenti sul punto P sono il peso \( m g\) e la tensione del filo \(T_F\) per cui il moto è regolato da \(m \overrightarrow{g} + \overrightarrow{T_F} = m \overrightarrow{a}\).
Considerando le componenti lungo la traiettoria e ortogonali a quest'ultima:
\begin{equation*}
    R_T = -mg \sin \theta = m a_T \ , \ R_N = T_F - mg \cos \theta = m a_N .
\end{equation*}
Il segno negativo della componente lungo la traiettoria è dovuto al fatto che  la forza ha segno opposto rispetto a quello della coordinata.
\(R_T\) è una forza di richiamo che tende a riportare il punto sulla verticale, anche se non è di direzione costante come nel caso delle forze elastiche.\newline
Ho quindi che \(a_T = L \frac {d^2 \theta} {dt^2}\) e \(a_N = \frac {v^2}{L}\), otteniamo:
\begin{equation*}
    \frac{d^{2} \theta}{d t^{2}}=-\frac{g}{L} \sin \theta \quad, \quad m \frac{v^{2}}{L}=T_{F}-m g \cos \theta
\end{equation*}
La prima è l'equazione differenziale del moto del pendolo, la sua soluzione è analiticamente complessa.
So però che per piccoli valori di \(\theta\):
\begin{equation*}
    \sin \theta = \theta - \frac {\theta^3} {3!} + ...
\end{equation*}
Quindi per piccole oscillazioni l'equazione differenziale diventa
\begin{equation*}
    \frac{d^2 \theta}{dt^2} + \frac{g}{L} \theta = 0
\end{equation*}
che posto \(\omega^2 = g/L\).\newline
La legge oraria del moto è:
\begin{equation}
    \theta = \theta_0 \sin (\omega t + \phi) ; 
\end{equation}
l'ampiezza \(\theta_0\) dell'oscillazione e la fase iniziale \(\phi\) dipendono dalle condizioni iniziali del moto.\newline
Il periodo \(T\) è dato da:
\begin{equation}
    T = \frac{2 \pi}{ \omega} = 2 \pi \sqrt{\frac{L}{g}}
\end{equation}
ed è indipendente dall'ampiezza.\newline
La legge oraria dello spostamento lungo l'arco di circonferenza è dato da
\begin{equation*}
    s = L \theta = L \theta_0 \sin (\omega t + \phi).
\end{equation*}
La velocità angolare e lineare diventano:
\begin{equation}
    \omega (t) = \frac{d \omega}{dt} = \omega \theta_0 \cos (\omega t + \phi)
\end{equation}
\begin{equation}
    v = \frac{d s}{dt} = L \frac{d \theta}{ dt} = L \omega \theta_0 \cos (\omega t +\phi).
\end{equation}
La velocità è massima quando il punto passa per la verticale \((\theta = 0)\) e nulla agli estremi delle oscillazioni \((\theta= \theta_0)\) dove il verso del molo si inverte.\newline
Quando l'ampiezza delle oscillazioni non è piccola il moto è ancora periodico, ma non armonico, e il periodo \(T'\) dipende dall'ampiezza.\newline
La tensione del filo che sostiene il punto è:
\begin{equation}
    T_F = m [ g \cos \theta(t) + \frac{v^2(t)}{L} ]
\end{equation}
La tensione è massima nella posizione verticale, dove sia \(cos  (\theta (t))\) che \(v(t)\) assumono i valori massimi, ed è minima ne i punti di inversione (indipendentemente dall'ampiezza). 

È importante fare la differenza tra il pendolo semplice che abbiamo analizzato ora e il pendolo conico.
Il moto del pendolo conico si svolge in un piano orizzontale ed è circolare uniforme; le forze sono ortogonali alla traiettoria e costanti ed è necessario comunicare al punto una velocità iniziale appropriata, che poi resta costante in modulo. 
Il moto del pendolo semplice si svolge in un piano verticale e la risultante delle forze ha sia componente tangente che normale alla traiettoria, entrambe non costanti; il moto può avvenire anche con velocità iniziale nulla, purché sia \(\theta \neq 0\), ed è armonico semplice lungo un arco di circonferenza.

\section{Tensione dei fili}

Nel pendolo, il filo di sostegno serve per applicare una certa forza al punto in movimento: il filo risulta teso e la forza, con direzione lungo il filo, che questo esercita sul punto viene chiamata \emph{tensione del filo}. Il filo può essere fissato in un estremo ad un punto fisso e nell'altro ad un punto materiale oppure può collegare due punti materiali.\newline
Per chiarire il concetto di tensione consideriamo un filo teso in quiete e prendiamo in esame un elemento infinitesimo di esso. Tale elemento è tirato dalle due parti restanti di filo e l'equilibrio statico richiede che le due forze, agenti sull'elemento di filo, siano eguali in modulo e direzione e di verso opposto.\newline
Ciò vale per qualunque elemento di filo e il valore della tensione è lo stesso ovunque. In particolare ad un estremo \(T = -F\).

Se il filo teso è in movimento il prodotto \(m_F a\) è nullo dato che consideriamo trascurabile la massa del filo; pertanto il valore della tensione è ancora lo stesso in qualunque punto del filo.

Riassumendo, il filo teso esercita agli estremi la tensione \(T\), il cui valore dipende dalle forze applicate, ma la reazione sul filo non può superare, per un filo reale, un valore massimo \(T_{MAX}\) , oltre il quale il filo si spezza.\newline
Non è necessario che il filo sia completamente rettilineo, esso può essere parzialmente avvolto attorno ad un disco (carrucola), con lo scopo di cambiare la direzione della forza. 

Nel risolvere gli esercizi sulla tensione dei fili è bene considerare un corpo alla volta, considerare le eventuali forze a cui è soggetto e metterle in ugualianza con la tensione \(T\).
Per trovare l'accelerazione del sistema di punti materiali legati dai fili tesi bisogna usare la legge di Newton ed e mettere a sistema nel caso ci siano più corpi (e quindi più vincoli).

\section{Lavoro, Potenza, Energia Cinetica}

\subsection{Lavoro}

Dato un punto materiale che si muove lungo una traiettoria curvilinea sollo l'azione di una forza \(\overrightarrow{F}\), si definisce lavoro della forza \(F\), compiuto durante lo spostamento del punto dalla posizione \(A\) a \(B\), la quantità scalare
\begin{equation}
    W = \int_A^B \overrightarrow{F} \cdot d \overrightarrow{s} = \int_A^B F \cos \theta ds = \int_A^B F_T ds . 
\end{equation} 
Il lavoro è l'integrale di linea della forza, cioè è dato dalla somma di infiniti contributi infinitesimi \(d W = F \cdot d s = F_T d s\).
Si osservi che in generale lungo la traiettoria sia \(\overrightarrow{F}\) che \(\theta\) sono variabili.

Una forza ortogonale alla traiettoria non compie lavoro perché \(\theta = \pi / 2 \). Il lavoro è positivo se \(0 \leq \theta < \pi / 2\) e in tale caso si parla di \emph{lavoro motore}. 
Il lavoro risulta negativo, \emph{lavoro resistente}, se \(\pi / 2 < \theta \leq \pi \). Lavoro resistente è compiuto ad esempio dalle forze di attrito, il cui verso è opposto allo spostamento, \(\theta = \pi\). 

Se sul punto P agiscono più forze per ciascuna posso calcolare il corrispondente lavoro e quindi il valoro complessivo risulta essere:
\begin{equation}
    W = \int_A^B \overrightarrow{F} \cdot d \overrightarrow{s} = \int_A^B (\overrightarrow{F_1} + ... + \overrightarrow{F_n}) \cdot d \overrightarrow{s} = \int_A^B \overrightarrow{F_1} \cdot d \overrightarrow{s} + ... + \int_A^B \overrightarrow{F_n} \cdot d \overrightarrow{s} = W_1 + ... W_n.
\end{equation}
quindi il lavoro è pari alla somma dei lavori delle singole forze agenti, ciascuno dei quali può essere positivo, negativo o nullo.

Il lavoro totale è nullo, \(W = 0\) quando non agisce nessuna forza oppure agiscono forze la cui risultante è nulla o è sempre ortogonale alla traiettoria (equilibrio dinamico).

\subsection{Potenza}

La potenza corrisponde al \emph{lavoro per unità di tempo}:
\begin{equation}
    P = \frac{dW}{dt} = \overrightarrow{F} \cdot \frac{d \overrightarrow{r}}{dt} = \overrightarrow{F} \cdot \overrightarrow{v} = F_T v .
\end{equation}
Questa è la potenza istantanea, in generale variabile durante il moto, e caratterizza la rapidità di erogazione del lavoro. 
La potenza media è il rapporto \(W/t\), cioè il lavoro totale diviso per il tempo durante il lavoro è stato svolto.
Questa grandezza è utile a quantificare le prestanzioni di un dispositivo o macchina che fornisce lavoro. Infatti se confronto due macchine posso sapere quale mi eroga la stessa quantitià di lavoro nel minor tempo tramite la potenza.

\subsection{Energia cinetica}

Ricordando la definizione di lavoro infinitesimo ad uno spostamento infinitesimo.
\begin{equation*}
    d W=F_T d s=m a_T d s=m \frac{d v}{d t} d s=m \frac{d s}{d t} d v=m v d v
\end{equation*}
Se considero un percorso dalla posizione A a B abbiamo:
\begin{equation}
    W=\int_{A}^{f} m v d v=\frac{1}{2} m v_{B}^{2}-\frac{1}{2} m v_{A}^{2}=E_{k, B}-E_{k, A}=\Delta E_{k}
\end{equation}
Il lavoro è eguale alla variazione della quantità \(\frac{1}{2} m v^2\) che si chiama energia cinetica del punto materiale.

Tutte le leggi con cui vengono definite le varie forme di energia contengono sempre la variazione di energia e pertanto tali quantità possono essere definite a meno di una costante. 
Per esempio l'energia cinetica di un punto potrebbe essere scritta come \(\frac{1}{2} m v^2 + costante\), senza modificare la l'equazione di definizione in quanto nella differenza la costante scompare.

Il lavoro è la manifestazione dell'azione di una forza ed è quindi conseguenza dell'interazione con l'ambiente circostante. 
Si parla pertanto di lavoro \emph{scambiato} e non si dice mai che  un s istema possiede lavoro. 
Si parla invece di energia posseduta dal sistema, che viene modificata dall'interazione con l'ambiente esterno. Un effetto misurabile dell'interazione è la variazione di energia. 

\section{Lavoro della forza peso}

Il lavoro della forza peso per uno spostamento da una posizione \(A\) a \(B\) è dato da:
\begin{equation}
    W = \int_A^B \overrightarrow{F} \cdot d \overrightarrow{s} = \overrightarrow{F} \cdot \int_A^B d \overrightarrow{s} = m \overrightarrow{g} \cdot \overrightarrow{r}_{AB}
\end{equation}
dove \(\overrightarrow{r}_{AB}\) è la distanza tra \(A\) e \(B\).

Svolgendo il prodotto scalare ottengo:
\begin{equation}
    W = - (mgz_b - mgz_a) =- (E_{p,B} - E_{p,A}) = - \Delta E_p
\end{equation}
Con \(E_p= m g z\) indichiamo una funzione della coordinata del punto \(z\) (misurata lungo un asse parallelo e di verso opposto alla forza peso) che ha questa proprietà: il lavoro è eguale all'opposto della variazione di questa funzione durante lo spostamento da \(A\) a \(B\) e pertanto non dipende dalla particolare traiettoria che collega i due punti (conservazione della forza).\newline
In base alla posizione di \(A\) e \(B\) posso trovarmi in due casi:
\begin{itemize}
    \item Se \(B\) si trova più in basso di \(A\), ho \(W>0\) quindi \(E_p\) diminuisce.
    \item Se \(B\) si trovia più in alto di \(A\), ho \(W<0\) quindi \(E_p\) aumenta 
    (il punto quindi deve avere sufficiente velocità iniziale così che la diminuzione di energia cinetica eguagli il lavoro oppure bisogna applicare al punto un'altra forza il cui lavoro motore superi in modulo il lavoro resistente della forza peso).
\end{itemize}

La trattazione fatta per la forza peso è applicabile a qualsiasi altra forza costante \(\overrightarrow{F}\): si prende un asse parallelo e discorde a \(\overrightarrow{F}\) utilizzano le stesse formule precedenti con \(F\) al posto di \(mg\).

\section{Lavoro di una forza elastica}

Il lavoro della forza elastica \(\overrightarrow{F} =-k x \hat{x}\), per uno spostamento lungo l'asse \(x\) vale:
\begin{equation}
    W = \int_A^B -k x\hat{x} \cdot dx \hat{x} = -k \int_A^B x dx = \frac{1}{2} k x^2_A - \frac{1}{2} k x^2_B = - \Delta E_p,
\end{equation}
con \(E_p = \frac{1}{2} k x^2\) in funzione della posizione.\newline
Come nel caso della forza peso si tratta di una forza conservativa, quindi non conta la traiettoria.
A seconda del punto di inizio e fine posso avere due casi:
\begin{itemize}
    \item Se la coordinata iniziale è maggiore di quella finale, cioè se il punto si muove verso il centro della forza, \(W>0\), quindi \(E_p\) diminuisce (spostamento naturale).
    \item Nel caso contrario di allontanamento dal centro \(W < 0\), \(E_p\) aumenta: per eseguire tale spostamento il punto deve possedere una velocità iniziale oppure si deve applicare una forza opportuna
\end{itemize}

\section{Lavoro di una forza di attrito radente}

Il lavoro di una forza di attrito radente è dato dall'equazione:
\begin{equation}
    W = \int_A^B \overrightarrow{F}_a \cdot d \overrightarrow{s} = \int_A^B -\mu_d N \hat{x} \cdot d \overrightarrow{s} = - \mu_d N \int_A^B ds .
\end{equation}
dove \(\hat{x}\) è il versore in direzione dello spostamento e \(\int_A^B\) è la lunghezza del percorso tra \(A\) e \(B\), misurata lungo \emph{la traiettoria effettiva del punto materiale}.\newline
Dato che il lavoro dipende dal percorso non si tratta di una forza conservativa.\newline
Il lavoro è sempre negativo, cioè è lavoro resistente.

Perché possa verificarsi il moto o deve agire un'altra forza che produca un lavoro motore oppure, in assenza di questa, il punto deve possedere una certa velocità iniziale, ovvero una certa energia cinetica \(E_{k,A}\).

Quindi l'energia cinetica iniziale diminuisce lungo il percorso quindi arriverò alla posizione \(B\) con una velocità minore.\newline
In particolare, data \(E_{k,A}\), il punto si ferma dopo un percorso:
\begin{equation*}
    s_{AB} = E_{k,A} / \mu_d N
\end{equation*} 


\end{document}