\documentclass[class=book, crop=false, oneside, 12pt]{standalone}
\usepackage{standalone}
\usepackage{amsmath}
\usepackage{../../style}
\graphicspath{{./assets/images/}}

% arara: pdflatex: { synctex: yes, shell: yes }
% arara: latexmk: { clean: partial }
\begin{document}

\chapter{Corrente elettrica}

\section{Conduzione elettrica}

I materiali conduttori solidi sono costituiti da un reticolo spaziale ai cui vertici si trovano gli ioni positivi (atomi che hanno perso uno o più elettroni) e al cui interno si muovono gli elettroni liberi. 
In un metallo questi sono gli unici portatori mobili di carica. 
Nel rame e nell'argento, in cui c'è un elettrone libero per atomo, il numero di elettroni per unità di volume coincide con il numero di atomi per unità di volume e abbiamo rispettivamente
\begin{equation*}
    n = \frac{N_A \rho}{A} = \frac{6.022 \cdot 10^{26} \cdot 8.96 \cdot 10^3}{63.55} = 8.49 \cdot 10^{28} \text{ elettroni / }m^3
\end{equation*}
\begin{equation*}
    n = \frac{6.022 \cdot 10^{26} \cdot 10.5 \cdot 10^3}{107.87} = 5.86 \cdot 10^{28} \text{ elettroni / }m^3
\end{equation*}

In qualsiasi volume \(\tau\), piccolo su scala macroscopica, ma contenente un numero \(N\) di elettroni abbastanza elevato, la velocità media è nulla: 
\begin{equation*}
    \overrightarrow{v}_m = \frac{1}{N} \sum_i \overrightarrow{v}_i = 0
\end{equation*}
indicando con \(\overrightarrow{v}_i\) le velocità dei singoli elettroni. 
Ciò vuol dire che non esiste una direzione di moto preferenziale per gli elettroni.

Se si mettono a contatto due conduttori \(C_1\) e \(C_2\) isolati, a potenziali \(V_1\) e \(V_2\) diversi, si raggiunge una condizione di equilibrio in cui entrambi i conduttori si portano allo stesso potenziale \(V\). 
Nel processo un certo numero di elettroni passa dal conduttore a potenziale minore a quello a potenziale maggiore, sotto l'azione del campo elettrico \(\overrightarrow{E}\) dovuto alla differenza di potenziale, \(\Delta V\). 
Questo moto ordinato di elettroni in una certa direzione costituisce una \emph{corrente elettrica} e il fenomeno è un esempio di \emph{conduzione elettrica}.

La corrente elettrica in questo caso specifico dura soltanto un tempo molto breve che impedisce l'esecuzione di studi sistematici del fenomeno. 

A tale scopo è necessario un dispositivo capace di mantenere una differenza di potenziale, e quindi un campo elettrico, tra due conduttori a contatto ovvero tra due punti di uno stesso conduttore. 
Così facendo il flusso di elettroni può durare per molto tempo e quindi nel conduttore si instaura una corrente elettrica stabile, in un regime di equilibrio dinamico e non più di equilibrio elettrostatico.

Un qualsiasi dispositivo con le caratteristiche appena descritte è definito come \emph{generatore di forza elettromotrice (f.e.m.)}.


\end{document}