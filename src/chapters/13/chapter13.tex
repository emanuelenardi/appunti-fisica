\documentclass[class=book, crop=false, oneside, 12pt]{standalone}
\usepackage{standalone}
\usepackage{amsmath}
\usepackage{../../style}
\graphicspath{{./assets/images/}}

% arara: pdflatex: { synctex: yes, shell: yes }
% arara: latexmk: { clean: partial }
\begin{document}

\chapter{Corrente elettrica}

\section{Conduzione elettrica}

I materiali conduttori solidi sono costituiti da un reticolo spaziale ai cui vertici si trovano gli ioni positivi (atomi che hanno perso uno o più elettroni) e al cui interno si muovono gli elettroni liberi. 
In un metallo questi sono gli unici portatori mobili di carica. 
Nel rame e nell'argento, in cui c'è un elettrone libero per atomo, il numero di elettroni per unità di volume coincide con il numero di atomi per unità di volume e abbiamo rispettivamente
\begin{equation*}
    n = \frac{N_A \rho}{A} = \frac{6.022 \cdot 10^{26} \cdot 8.96 \cdot 10^3}{63.55} = 8.49 \cdot 10^{28} \text{ elettroni / }m^3
\end{equation*}
\begin{equation*}
    n = \frac{6.022 \cdot 10^{26} \cdot 10.5 \cdot 10^3}{107.87} = 5.86 \cdot 10^{28} \text{ elettroni / }m^3
\end{equation*}

In qualsiasi volume \(\tau\), piccolo su scala macroscopica, ma contenente un numero \(N\) di elettroni abbastanza elevato, la velocità media è nulla: 
\begin{equation*}
    \overrightarrow{v}_m = \frac{1}{N} \sum_i \overrightarrow{v}_i = 0
\end{equation*}
indicando con \(\overrightarrow{v}_i\) le velocità dei singoli elettroni. 
Ciò vuol dire che non esiste una direzione di moto preferenziale per gli elettroni.

Se si mettono a contatto due conduttori \(C_1\) e \(C_2\) isolati, a potenziali \(V_1\) e \(V_2\) diversi, si raggiunge una condizione di equilibrio in cui entrambi i conduttori si portano allo stesso potenziale \(V\). 
Nel processo un certo numero di elettroni passa dal conduttore a potenziale minore a quello a potenziale maggiore, sotto l'azione del campo elettrico \(\overrightarrow{E}\) dovuto alla differenza di potenziale, \(\Delta V\). 
Questo moto ordinato di elettroni in una certa direzione costituisce una \emph{corrente elettrica} e il fenomeno è un esempio di \emph{conduzione elettrica}.

La corrente elettrica in questo caso specifico dura soltanto un tempo molto breve che impedisce l'esecuzione di studi sistematici del fenomeno. 

A tale scopo è necessario un dispositivo capace di mantenere una differenza di potenziale, e quindi un campo elettrico, tra due conduttori a contatto ovvero tra due punti di uno stesso conduttore. 
Così facendo il flusso di elettroni può durare per molto tempo e quindi nel conduttore si instaura una corrente elettrica stabile, in un regime di equilibrio dinamico e non più di equilibrio elettrostatico.

Un qualsiasi dispositivo con le caratteristiche appena descritte è definito come \emph{generatore di forza elettromotrice (f.e.m.)}.

\section{Corrente elettrica, corrente elettica}

Supponiamo che in una certa regione di un conduttore ci siano \(n\) portatori di carica \(+e\) per unità di volume e che in essa agisca un campo elettrico \(\overrightarrow{E}\) prodotto da un generatore di forza elettromotrice; i portatori si muovono sotto l'azione della forza elettrica \(\overrightarrow{F} = e \overrightarrow{E}\), acquistando, la velocità \(v_d\) lungo la direzione del campo elettrico \(\overrightarrow{E}\), detta velocità di deriva. 
Tale moto dà origine ad una corrente elettrica.

Consideriamo una superficie \(\Sigma\) tracciata all'interno del conduttore: detta \(\Delta q\) la carica che passa nel tempo \(\Delta t\) attraverso la superficie si definisce intensità di corrente media la grandezza:
\begin{equation*}
    i_m = \frac{\Delta q}{\Delta t}
\end{equation*}

La intensità di corrente istantanea è definita come il limite per \(\Delta t \rightarrow 0\) della intensità media:
\begin{equation*}
    i = \lim_{\Delta t \rightarrow 0 } \frac{\Delta q}{\Delta t} = \frac{dq}{dt}
\end{equation*} 

Per mettere in relazione la corrente elettrica con il moto delle cariche ci riferiamo a una superficie infinitesima \(d\Sigma\): la cui normale \(\overrightarrow{u}_n\) formi un angolo e con il campo elettrico \(\overrightarrow{E}\) e quindi con la velocità \(\overrightarrow{v}_d\) delle cariche positive. 
Nel tempo \(\Delta t\) le cariche percorrono la distanza \(v_d \Delta t\) per cui la carica complessiva che passa attraverso \(d\Sigma\): nel tempo \(\Delta t\) è quella contenuta nel volume infinitesimo \(d \tau\) definito da \(d \Sigma\): e \(v_d \Delta t\):
\begin{equation*}
    \Delta \tau = v_d \Delta t d \Sigma \cos \theta \ , \ \Delta q = n_{+} e d \tau = n_{+} e v_d d \Sigma \cos \theta \Delta t
\end{equation*}

La carica che passa nell'unità di tempo attraverso \(d \Sigma\):, cioè l'intensità di corrente attraverso \(d \Sigma\):
\begin{equation*}
    di = n_{+} e v_d d \Sigma \cos \theta
\end{equation*}

Definiamo la densità di corrente \(j\) come
\begin{equation}
    \overrightarrow{j} = n_{+} e \overrightarrow{v}_d
\end{equation}
e riscriviamo 
\begin{equation} \label{intensita_infinitesima}
    di = \overrightarrow{j} \cdot \overrightarrow{u}_n d \Sigma
\end{equation}

L'intensità di corrente attraverso la superficie finita \(\Sigma\): si ottiene integrando (\ref{intensita_infinitesima})
\begin{equation} \label{intensita_di_corrente}
    i = \int_{\Sigma} \overrightarrow{j} \cdot \overrightarrow{u}_n d \Sigma
\end{equation}
essa \emph{risulta eguale al flusso del vettore densità di corrente attraverso la superficie} \(\Sigma\).

In particolare se la superficie \(\Sigma\): è ortogonale a \(\overrightarrow{j}\), cioè a \(\overrightarrow{v}_d\) e \(\overrightarrow{j}\) ha lo stesso valore in tutti i punti di \(\Sigma\),
\begin{equation}
    i = j \Sigma \ , \ j = \frac{i}{\Sigma}
\end{equation}
\emph{la densità di corrente è la corrente che attraversa l'unità di superficie perpendicolare alla direzione del moto delle cariche ( e resta così giustificato il nome densità di corrente). }

Se, come nei conduttori metallici, i portatori di carica sono negativi, fissata la direzione e il verso di \(\overrightarrow{E}\) la velocità di deriva \(\overrightarrow{v}_{-}\) è diretta in verso opposto rispetto al campo elettrico. 
Il vettore \(-e v_{-}\) ha invece lo stesso verso di \(\overrightarrow{E}\) e la densità di corrente, detto \(v_{-}\) il numero di portatori per unità di volume, è
\begin{equation}
    \overrightarrow{j} = -n_{-} e \overrightarrow{v}_{-}
\end{equation}
\emph{parallela e concorde } al campo elettrico.

Che la densità di corrente sia sempre concorde a \(\overrightarrow{E}\), discende dalla definizione di \(\overrightarrow{j}\) come prodotto della carica per unità di volume ( con il suo segno) per la velocità di deriva e riflette la circostanza sperimentale che su scala macroscopica non è possibile correlare il verso della corrente al segno dei portatori di carica: fissata una data differenza di potenziale gli stessi effetti si hanno se la conduzione è dovuta a cariche positive con moto concorde a \(\overrightarrow{E}\) oppure a cariche negative con moto discorde a \(\overrightarrow{E}\). 

In base a queste considerazioni si assume convenzionalmente come verso della corrente quello del moto delle cariche positive, ovvero quello che va dai punti a potenziale maggiore ai punti a potenziale minore.

\subsection{Corrente elettrica stazionaria}
%TODO add image
Consideriamo un conduttore percorso da una corrente di densità \(\overrightarrow{j}\). 
Se \(\Sigma_1\) e \(\Sigma_2\) sono due diverse sezioni del conduttore, le intensità di corrente attraverso le due sezioni sono, in base alla (\ref{intensita_di_corrente}),
\begin{equation*}
    i_1 = \int_{\Sigma_1} \overrightarrow{j}_1 \cdot \overrightarrow{u}_1 d \Sigma_1 \ , \ i_2 = \int_{\Sigma_2} \overrightarrow{j}_2 \cdot \overrightarrow{u}_2 d \Sigma_2
\end{equation*}
e rappresentano rispettivamente la carica che \emph{entra} e la carica che \emph{esce} nell'unità di tempo nel volume delimitato da \(\Sigma_1, \Sigma_2\) e dalla superficie laterale \(\Sigma_1\) attraverso la quale non c'è flusso di carica. 
Se si fa l'ipotesi che nell'interno del tronco di cono di basi \(\Sigma_1, \Sigma_2\) non vari nel tempo la carica, allora: 
\begin{equation}
    i_1 = i_2
\end{equation}
Questa condizione è detta di stazionarietà: \emph{in condizioni stazionarie l'intensità di corrente è costante attraverso ogni sezione del conduttore}. 

\end{document}