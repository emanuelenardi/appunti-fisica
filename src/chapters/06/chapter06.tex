\documentclass[class=book, crop=false, oneside, 12pt]{standalone}
\usepackage{standalone}
\usepackage{amsmath}
\usepackage{../../style}
% \graphicspath{{./assets/images/}}

% arara: pdflatex: { synctex: yes, shell: yes }
% arara: latexmk: { clean: partial }
\begin{document}

\chapter{Primo principio della termodinamica}

\section{Sistemi e stati termodinamici}

\subsection{Introduzione}

Un sistema termodinamico è spesso assimilabile, da un punto di vista meccanico, ad un sistema continuo, considerato che microscopicamente è costituito da un numero di elementi dell'ordine del numero di Avogadro, \(N_A = 6.022 \cdot 10^{23}\).
Cercheremo invece di descrivere le trasformazioni che il sistema può subire e gli scambi energetici che ne risultano con l'ambiente circostante, individuando le grandezze più appropriate a tale descrizione.

\subsection{Sistema termodinamico}

Chiamiamo sistema termodinamico una porzione del mondo che può essere costituita da una o più parti, per esempio un volume di gas, un liquido in equilibrio, con il suo vapore, un insieme di blocchi di solidi diversi; tale sistema è oggetto delle nostre osservazioni per quanto riguarda le proprietà fisiche macroscopiche che lo caratterizzano e le loro eventuali variazioni. 

\subsection{Ambiente e Universo}

Per ambiente circostante, o semplicemente ambiente, intendiamo quell'insieme che può essere costituito da una sola parte (per esempio l'aria o un altro fluido in cui è immerso il sistema) o da più parti (per esempio diversi corpi solidi a contatto con il sistema), con cui il sistema può interagire: l'ambiente pertanto contribuisce in generale a determinare le caratteristiche fisiche macroscopiche del sistema e la loro evoluzione.\newline
L'insieme sistema più ambiente si chiama universo termodinamico, in senso locale.

\subsection{Sistema aperto}

Se tra il sistema e l'ambiente avvengono scambi di energia e di materia il sistema è detto aperto. 
Ad esempio, se il sistema è costituito da un liquido in ebollizione e l'ambiente dal recipiente che contiene il liquido, dall'atmosfera esterna compreso il vapore e dalla sorgente di calore, nel processo di ebollizione si ha trasformazione di liquido in vapore e quindi passaggio di materia dal sistema all'ambiente; 
inoltre vi è certamente passaggio di energia dall'ambiente al sistema tramite la sorgente di calore.

\subsection{Sistema chiuso}

Il sistema si dice chiuso se sono esclusi scambi di materia, ma si hanno solamente scambi di energia. 
Ritornando all'esempio precedente, il liquido è contenuto in un recipiente chiuso, a contatto con la sorgente di calore; il vapore prodotto rimane all'interno del sistema. 

\subsection{Sistema isolato}

Infine il sistema è detto isolato se non avvengono scambi di energia e di materia con un altro sistema esterno, cioè con l'ambiente. 
L'universo termodinamico formato da un sistema e dal suo ambiente è da considerarsi come un sistema isolato. 

\subsection{Variabili termodinamiche}

Un sistema termodinamico viene descritto tramite un numero ridotto di grandezze fisiche direttamente misurabili, dette coordinate o variabili termodinamiche, come volume, pressione, temperatura (che definiremo tra breve), massa, concentrazione, densità, ecc. 

Alcune variabili termodinamiche, chiamate variabili estensive, sono additive, come massa e volume; altre invece chiamate variabili intensive, dipendono in generale dalla posizione del punto nel sistema, e non sono additive.

Il numero minimo di coordinate termodinamiche necessario per descrivere completamente un sistema termodinamico non è fissato a priori, ma dipende dalle caratteristiche chimico-fisiche dei vari sistemi che vengono studiati. 
Le proprietà di un sistema vengono sempre espresse in funzione dei valori delle sue coordinate termodinamiche. 

Osserviamo che la definizione di stato termodinamico è concettualmente diversa da quella di stato meccanico, per il quale in linea di principio si presuppone la conoscenza di posizione e velocità di ciascuno degli \(n\) punti che costituiscono il sistema. 
In questi termini un sistema termodinamico non è definibile, visto il grande valore di \(n\). 
In effetti, se è noto lo stato termodinamico, non è noto in generale quello meccanico, anzi a un dato stato termodinamico possono corrispondere moltissimi stati meccanici diversi. 


\end{document}