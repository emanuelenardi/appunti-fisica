\documentclass[class=book, crop=false, oneside, 12pt]{standalone}
\usepackage{standalone}
\usepackage{amsmath}
\usepackage{../../style}
% \graphicspath{{./assets/images/}}

% arara: pdflatex: { synctex: yes, shell: yes }
% arara: latexmk: { clean: partial }
\begin{document}

\chapter{Primo principio della termodinamica}

\section{Sistemi e stati termodinamici}

\subsection{Introduzione}

Un sistema termodinamico è spesso assimilabile, da un punto di vista meccanico, ad un sistema continuo, considerato che microscopicamente è costituito da un numero di elementi dell'ordine del \emph{numero di Avogadro}, \(N_A = 6.022 \cdot 10^{23}\).
Cercheremo invece di descrivere le trasformazioni che il sistema può subire e gli scambi energetici che ne risultano con l'ambiente circostante, individuando le grandezze più appropriate a tale descrizione.

\subsection{Sistema termodinamico}

Chiamiamo sistema termodinamico una porzione del mondo che può essere costituita da una o più parti, per esempio un volume di gas, un liquido in equilibrio, con il suo vapore, un insieme di blocchi di solidi diversi; tale sistema è oggetto delle nostre osservazioni per quanto riguarda le proprietà fisiche macroscopiche che lo caratterizzano e le loro eventuali variazioni. 

\subsection{Ambiente e Universo}

Per ambiente circostante, o semplicemente ambiente, intendiamo quell'insieme che può essere costituito da una sola parte (per esempio l'aria o un altro fluido in cui è immerso il sistema) o da più parti (per esempio diversi corpi solidi a contatto con il sistema), con cui il sistema può interagire: l'ambiente pertanto contribuisce in generale a determinare le caratteristiche fisiche macroscopiche del sistema e la loro evoluzione.\newline
L'insieme sistema più ambiente si chiama universo termodinamico, in senso locale.

\subsection{Sistema aperto}

Se tra il sistema e l'ambiente avvengono scambi di energia e di materia il sistema è detto aperto. 
Ad esempio, se il sistema è costituito da un liquido in ebollizione e l'ambiente dal recipiente che contiene il liquido, dall'atmosfera esterna compreso il vapore e dalla sorgente di calore, nel processo di ebollizione si ha trasformazione di liquido in vapore e quindi passaggio di materia dal sistema all'ambiente; 
inoltre vi è certamente passaggio di energia dall'ambiente al sistema tramite la sorgente di calore.

\subsection{Sistema chiuso}

Il sistema si dice chiuso se sono esclusi scambi di materia, ma si hanno solamente scambi di energia. 
Ritornando all'esempio precedente, il liquido è contenuto in un recipiente chiuso, a contatto con la sorgente di calore; il vapore prodotto rimane all'interno del sistema. 

\subsection{Sistema isolato}

Infine il sistema è detto isolato se non avvengono scambi di energia e di materia con un altro sistema esterno, cioè con l'ambiente. 
L'universo termodinamico formato da un sistema e dal suo ambiente è da considerarsi come un sistema isolato. 

\subsection{Variabili termodinamiche}

Un sistema termodinamico viene descritto tramite un numero ridotto di grandezze fisiche direttamente misurabili, dette coordinate o variabili termodinamiche, come volume, pressione, temperatura, massa, concentrazione, densità, ecc. 

Alcune variabili termodinamiche, chiamate variabili estensive, sono additive, come massa e volume; altre invece chiamate variabili intensive, dipendono in generale dalla posizione del punto nel sistema, e non sono additive.

Il numero minimo di coordinate termodinamiche necessario per descrivere completamente un sistema termodinamico non è fissato a priori, ma dipende dalle caratteristiche chimico-fisiche dei vari sistemi che vengono studiati. 
Le proprietà di un sistema vengono sempre espresse in funzione dei valori delle sue coordinate termodinamiche. 

Osserviamo che la definizione di stato termodinamico è concettualmente diversa da quella di stato meccanico, per il quale in linea di principio si presuppone la conoscenza di posizione e velocità di ciascuno degli \(n\) punti che costituiscono il sistema. 
In questi termini un sistema termodinamico non è definibile, visto il grande valore di \(n\). 
In effetti, se è noto lo stato termodinamico, non è noto in generale quello meccanico, anzi a un dato stato termodinamico possono corrispondere moltissimi stati meccanici diversi. 

\section{Equilibrio termodinamico, principio dell'equilibrio termico}

Lo stato termodinamico di un sistema è detto di equilibrio quando le variabili termodinamiche che lo descrivono sono costanti nel tempo. 
In un sistema termodinamico all'equilibrio le variabili termodinamiche sono dette variabili di stato.

\subsection{Equilibrio termodinamico}

L'equilibrio termodinamico è il risultato di tre diversi tipi di equilibrio, che devono essere realizzati contemporaneamente:

\begin{itemize}
    \item \emph{parete diatermica}, inteso come equilibrio di forze e momenti, secondo quanto studiato in meccanica; 
    \item \emph{equilibrio chimico}: non avvengono reazioni chimiche o trasferimenti di un componente del sistema entro il sistema stesso; 
    \item \emph{equilibrio termico}: la temperatura è la stessa ovunque.
\end{itemize}

Se uno stato è di equilibrio, le condizioni di equilibrio devono essere soddisfatte all'interno del sistema o di ciascuna delle sue parti, nell'interazione tra le parti del sistema e in quella tra sistema e ambiente. 
Quando c'è equilibrio con l'ambiente, vuol dire che esiste equilibrio tra le forze macroscopiche, qualunque sia la loro natura, agenti dall'esterno sul sistema e quelle sviluppate dal sistema; 
inoltre la temperatura del sistema, se questo non è isolato termicamente, è eguale alla temperatura dell'ambiente. 

In uno stato di equilibrio esiste, in generale, una precisa relazione tra le coordinate termodinamiche che si esprime sotto forma di equazione di stato.

\subsection{Trasformazione termodinamica}

Dati due diversi stati di equilibrio termodinamico di un certo sistema, l'eventuale evoluzione del sistema dal primo al secondo stato, spontanea o per effetto dell'interazione con l'ambiente, si chiama trasformazione termodinamica del sistema. 
Considereremo sempre come stati iniziali e finali di una certa trasformazione stati di equilibrio. 
Invece gli stati intermedi attraverso cui passa il sistema durante l'evoluzione possono essere di equilibrio o di non equilibrio; in questo secondo caso non è detto che si possano determinare tutte le coordinate termodinamiche del sistema. 
Ai fini del calcolo si considerano anche trasformazioni infinitesime, tra stati molto prossimi, le cui coordinate differiscono di quantità infinitesime.

\subsection{Equilibrio termico}

Si considerino due sistemi \(A\) e \(B\), ciascuno in equilibrio termodinamico, con il sistema \(A\) alla temperatura \(T_A\) e quello \(B\) alla temperatura \(T_B\).
I sistemi si dicono in equilibrio termico tra loro quando hanno la stessa temperatura, \(T_A = T_B\): la temperatura è pertanto l'indice del'equilibrio termico tra due sistemi. 

È verificato sperimentalmente il seguente principio dell'equilibrio termico: due sistemi che siano ciascuno in equilibrio termico con un terzo sistema sono in equilibrio termico tra loro. 
Se il sistema \(A\) è in equilibrio termico con il sistema \(C\) (\(T_A = T_C\)) e se anche il sistema \(B\) è in equilibrio termico con \(C\) (\(T_B = T_C\)), allora \(A\) è in equilibrio termico con \(B\) ( \(T_A = T_B\)).

Un metodo possibile per portare due sistemi all'equilibrio termico è quello di tenerli a contatto, tramite una parete. 
Se viene raggiunto l'equilibrio termico si parla di \emph{parete diatermica}, mentre se non si raggiunge mai l'equilibrio termico, e pertanto le due temperature sono indipendenti, la parete è detta \emph{adiabatica}. 
Nella realtà la situazione adiabatica è un caso limite, che può essere realizzato per tempi brevi, ma non in assoluto.

\subsection{Contatto termico}

Due sistemi separati da una parete diatermica si dicono anche in contatlo termico tra loro e inevitabilmente raggiungono l'equilibrio termico. 
Il contatto termico si può realizzare anche direttamente, senza alcuna parete, come avviene per due corpi solidi a contatto, per un corpo solido immerso in un fluido o per due fluidi non miscibili a contatto; la parete diatermica si rende necessaria quando bisogna conte-nere il sistema, come avviene nel caso di un gas. 

\subsection{Sistema adiabatico}

Un sistema è detto adiabatico se è circondato da pareti adiabatiche e quindi non può essere messo in contatto termico con un altro sistema o  con l'ambiente. 
Una parete è sempre necessaria per impedire o ritardare l'equilibrio termico. 

Notiamo infine che l'esistenza di equilibrio termico tra due sistemi non presuppone affatto che essi siano anche in altri tipi di equilibrio, ne viceversa.

\section{Definizione temperatura, termometri}

Per fare una definizione operativa di temperatura devono essere realizzate due condizioni. 
Deve esistere una grandezza \(X\) che caratterizza un fenomeno fisico e che varia con la temperatura. 
\(X\) si chiama \emph{caratteristica termometrica} e la temperatura è una funzione di \(X\), \(\theta (X)\), detta \emph{funzione termometrica}. 
Il dispositivo in cui avviene il fenomeno e che fornisce il valore della caratteristica termometrica è indicato come \emph{termometro}.

\subsection{Punto fisso}

In secondo luogo deve esistere un sistema, in uno stato di equilibrio, definibile con precisione e riproducibile con facilità, cui viene attribuito un valore arbitrario di temperatura, detto \emph{punto fisso}.

Il punto fisso campione, è il \emph{punto triplo} dell'acqua, ovvero quel particolare stato in cui ghiaccio, acqua e vapor d'acqua saturo sono in equilibrio. 
Al punto triplo dell'acqua è stata assegnata arbitrariamente la temperatura di \(273.16 K\), dove il simbolo \(K\) indica l'unità di misura prescelta, il kelvin.

\subsection{Misura della temperatura}

Per arrivare a esprimere numericamente la temperatura stabiliamo in via preliminare che la forma della funzione \(\theta(X)\) sia \(\theta(X) = aX\), con \(a\) costante.
È la scelta più semplice, giustificata però dal fatto che è valida per il termometro assoluto.

Il sistema, di cui vogliamo misurare la temperatura, viene messo a contatto termico con un termometro che, all'equilibrio termico, fornisce il valore \(X\).

Tale termometro al punto triplo dell'acqua dà il valore \(X_{pt}\) e per definizione abbiamo
\begin{equation*}
    \theta (X_{pt}) = a X_{pt} = 273.16
\end{equation*}
da cui \(a = 273.16/X_{pt}\), valore valido per quel termometro. Ne segue che la temperatura \(T\) del sistema, espressa dal valore della funzione \(\theta (X) = a X\), si scrive
\begin{equation}
    T = 273.16 \frac{X}{X_{pt}} K    
\end{equation}

L'ultima formula è la formula fondamentale per ogni termometro e fornisce la temperatura empirica di quel termometro.

Si usa il termine temperatura empirica in quanto, sperimentalmente, si constata che termometri di tipo diverso, o anche due diversi termometri dello stesso tipo, danno sempre letture diverse quando sono in equilibrio termico con lo stesso stato di un certo sistema, pur dando per costruzione tutti la stessa temperatura al punto triplo dell'acqua.
Se si vuole verificare se due sistemi sono alla stessa temperatura si deve metterli uno alla volta in contatto termico con lo stesso termometro. 

\subsection{Scale termometriche}

La scala che viene più comunemente usata nelle normali misure di temperatura è quella Celsius, in cui la temperatura del punto triplo dell'acqua vale \(0.01\) gradi Celsius (simbolo \(^{\circ} C\)).
Pertanto lo zero della scala Celsius è a \(273.15 K\) e corrisponde alla temperatura di fusione del ghiaccio a pressione atmosferica. 

Il valore di una differenza di temperatura è assunto lo stesso in gradi Celsius o in kelvin e pertanto la formula di conversione dal valore in kelvin \(T (K)\) al valore in gradi Celsius \(t ( ^{\circ} C)\) è semplicemente
\begin{equation*}
    t (^{\circ} C) = T (K) - 273.15
\end{equation*}

Nei paesi anglosassoni vengono utilizzate altre due scale di temperatura, la scala Rankine \(t ( ^{\circ} R)\) e la scala Fahrenheit \( t ( ^{\circ} F)\), che sono così definite rispetto alla temperatura espressa in kelvin:
\begin{equation*}
    t ( ^{\circ} R) = \frac{9}{5} T (K)
\end{equation*}
\begin{equation*}
    t (^{\circ} F) = \frac{9}{5} T (K) - 459.67
\end{equation*}

Il legame tra scala Fahrenheit e scala Celsius è pertanto:
\begin{equation*}
    t (^{\circ} F) = \frac{9}{5} t (^{\circ} C) + 32 \ , \ t(^{\circ} C) = \frac{5}{9} \left[t (^{\circ} F ) - 32\right]
\end{equation*}

Nella scala Fahrenheit il punto di fusione del ghiaccio (\(O ^{\circ} C \)) corrisponde a \(32 ^{\circ} F\) e il punto di ebollizione dell'acqua (\(100 ^{\circ} C\)) a \(212 ^{\circ} F\); la temperatura ambiente di \(20 ^{\circ} C\) vale \(68 ^{\circ} F\). 

\section{Sistemi adiabatici, esperimenti di Joule, Calore}

\subsection{Esperimenti di Joule}

Joule condusse una serie di esperimenti fondamentali sugli effetti termici del lavoro meccanico. 
Schematicamente, le varie esperienze eseguite da Joule su un sistema termodinamico costituito da una certa quantità d'acqua avevano lo scopo di realizzare un aumento della temperatura del sistema con procedimenti diversi. 

\begin{itemize}
    \item Viene messo in rotazione un mulinello nell'acqua spendendo il lavoro \(W_1\), fornito dalla variazione di energia potenziale di due masse che scendono sotto l'azione della forza di gravità. 
    Con varie palette fisse immerse nell'acqua si impedisce che essa entri in rotazione. 
    L'acqua, agitata dal mulinello, viene riscaldata per effetto dell'attrito. 
    \item Viene immerso nell'acqua un conduttore di resistenza \(R\), percorso da corrente. 
    \(W_2\) è il lavoro speso per fare circolare la corrente.
    \item Viene compressa una certa quantità di gas, contenuta in un recipiente con pareti diatermiche, immerso nell'acqua. 
    Il processo di compressione del gas richiede un lavoro.
    \item Vengono strofinati tra loro due blocchi di metallo immersi nell'acqua.  
    Il lavoro speso contro le forze di attrito è \(W_4\).
\end{itemize}
Nelle varie esperienze l'insieme costituito dall'acqua e dal dispositivo meccanico o elettrico è racchiuso entro pareti adiabatiche. 

Il risultato fondamentale osservato da Joule è che il lavoro speso a parità di massa d'acqua, \(W_1\) o \(W_2\) o \(W_3\) o \(W_4\), è sempre proporzionale alla variazione di temperatura dell'acqua con la stessa costante di proporzionalità.\newline
Il sistema termodinamico massa d'acqua passa da uno stato iniziale di equilibrio, caratterizzato dal valore \(T_{in}\) della temperatura, ad uno stato finale di equilibrio con temperatura \(T_{fin}\) tramite quattro diversi processi, ma il lavoro meccanico è sempre lo stesso.

Sulla base delle considerazioni per l'energia potenziale nel caso di forze conservative, scriviamo la seguente relazione:
\begin{equation}
    W_{ad} = - \Delta U = U_{in} - U_{fin}    
\end{equation}
dove \(U\) è una funzione che dipende solo dallo stato del sistema cioè dalle sue coordinate termodinamiche.

Se il sistema fornisce lavoro all'esterno, \(W\) è assunto positivo e pertanto l'energia \(U\) diminuisce; se invece si compie lavoro dall'esterno sul sistema \(W\) è assunto negativo e l'energia \(U\) aumenta.

Se possiamo ottenere lo stesso cambiamento di stato termodinamico dell'acqua, segnalato dalla stessa variazione di temperatura, tramite scambio di calore o di lavoro meccanico, possiamo postulare l'equivalenza degli effetti delle due procedure e scrivere anche nel caso di scambio di calore con lavoro nullo, una relazione analoga a:
\begin{equation}
    Q = \Delta U
\end{equation}
assumendo positivo il calore ceduto al sistema dall'esterno. 

\subsection{Equivalenza calore e lavoro}

Pertanto 
\begin{equation}
    Q = -W
\end{equation}
dove rappresenta il calore scambiato, senza lavoro esterno per far variare di \(\Delta T\) la tempratura della massa d'acqua e \(W\) il lavoro che deve essere speso, in condizioni adiabatiche, per ottenere la stessa variazione di temperatura. 

L'ultima equazione è detta \emph{equivalenza tra calore e lavoro}; essa indica anche come si possa eseguire una misura del calore scambiato. 
Il calore viene in questo modo espresso in joule.


\end{document}