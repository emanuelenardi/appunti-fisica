\documentclass[class=book, crop=false, oneside, 12pt]{standalone}
\usepackage{standalone}
\usepackage{amsmath}
\usepackage{../../style}
% \graphicspath{{./assets/images/}}

% arara: pdflatex: { synctex: yes, shell: yes }
% arara: latexmk: { clean: partial }
\begin{document}

\chapter{Forza elettrostatica. Campo elettrostatico}

\section{Cariche elettiche, isolanti e conduttori}

Tra le interazioni fondamentali esistenti in natura la prima ad essere scoperta e studiata quantitativamente è stata l'interazione gravitazionale, responsabile di gran parte dei fenomeni che si osservano su scala macroscopica nell'universo.
Essa è dettata dalla legge:
\begin{equation}
    F = \gamma \frac{m_1 m_2}{r^2}
\end{equation}

Un'altra interazione fondamentale, che gioca un ruolo essenziale nella costituzione della materia, è quella elettromagnetica. 
Un aspetto particolare dell'interazione elettromagnetica è la forza elettrica.

Oggi noi attribuiamo le forze in parola a cariche elettriche, che preesistono nei corpi e che passano da un corpo all'altro durante lo strofinio, per cui i corpi elettrizzati si chiamano anche elettricamente carichi.
Questi corpi che si caricano per strofinio sono detti isolanti, in quanto capaci di trattenere la carica elettrica, mentre altri, come ad esempio i metalli e il corpo umano stesso, non trattengono la carica e sono detti conduttori.

Sperimentalmente si deduce che esistono due diversi tipi di cariche elettriche; per convenzione è stata chiamata positiva la carica che compare sulla superficie delle sostanze tipo vetro quando vengono elettrizzate, mentre è stata chiamata negativa la carica che compare sulla superficie delle sostanze tipo bachelite.
Da questo ottengo i seguenti risultati:
\begin{itemize}
    \item due corpi isolanti carichi entrambi positivamente o entrambi negativa-mente si respingono; 
    \item un corpo isolante carico positivamente e uno carico negativamente si  at-traggono; 
    \item nel processo di carica per strofinio i due corpi, la bacchetta di isolante e il panno, acquistano sempre una carica di segno opposto. 
\end{itemize}

La carica che si accumula per strofinio sugli isolanti si mantiene per tempi considerevoli, specialmente se l'aria nell'ambiente in cui si opera è secca. 
Invece, come abbiamo già rilevato, non è possibile caricare per strofinio una bacchetta di metallo tenendola in mano, come si fa con le bacchette di isolante, a meno che non sia tenuta da una barretta di isolante, in tal caso si comporta in modo analogo.
L'assenza di elettrizzazione dei conduttori si spiega col fatto che i metalli e il corpo umano sono conduttori, cioè permettono il movimento della carica elettrica accumulatasi durante lo strofinio, a differenza di quanto avviene negli isolanti.
In questi esperimenti hanno caratteristiche di conduttori anche il suolo, svariati liquidi tra cui l'acqua e anche l'aria umida.


\end{document}